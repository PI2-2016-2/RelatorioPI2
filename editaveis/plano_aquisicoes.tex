\chapter{Plano de Gerenciamento de Aquisições}
	\label{plano_de_aquisicoes}    						
% 	\input{anexos/Plano_de_Gerenciamento_de_aquisicoes}

%%%%%%%%%%%%%%%%%%%%%%% PLANO DE GERENCIAMENTO DE AQUISIÇÕES

\begin{center}
 {\large Plano de gerenciamento de Aquisições}\\[0.2cm]
 {Bancada para Ensaios de Vibração Mecânica}\\
 \end{center}
 
 \section*{Histórico de Alterações}
\begin{table}[h]
\centering
\begin{tabular}{|c|c|p{6cm}|p{5cm}|}

Data & Versão & Descrição & Responsável\\
\hline                               
28/08/2016 & 1.0 & Criação do Plano de Gerenciamento de Aquisições & Emilie Morais\\
\hline
\end{tabular}
\end{table}

\section*{Objetivo}
  O objetivo desse plano é definir como será realizado o gerenciamento de aquisições do projeto.
  
\section*{Descrição do processo de gerenciamento das aquisições}
 A partir dos requisitos do projeto serão estabelecidos os itens necessários a serem adquiridos. Para cada produto será estipulada uma data na qual o produto deve estar disponível para o projeto. 
 
Considerando essa data, haverá a solicitação de aquisição pelo líder da área. Caso o item solicitado não tenha sido previsto no início do projeto a necessidade de aquisição será verificada. 

Após a aprovação da solicitação pelo grupo, o fornecedor será escolhido, o recurso financeiro será alocado e a aquisição será efetivada.

\section*{Gerenciamento e formas de aquisição}
As formas de aquisição dos produtos/serviços serão:
\begin{itemize}
	\item Compra imediata;
    \item Encomenda;
    \item Empréstimo;
    \item Contrato mensal;
\end{itemize}
Para os produtos de compra imediata e encomenda a compra será efetuada no valor total do produto e para o serviço de hospedagem do software será feito um plano gratuito de 3 meses para realização do projeto. Para uso do sistema após o término do projeto, a hospedagem passará a ter um valor fixo por mês.

\section*{Critérios de avaliação de cotações e propostas}
\label{criterios}
A escolha dos fornecedores será realizada considerando custo-benefício do produto ou serviço e a viabilidade de entrega no prazo necessário para o projeto.   

\section*{Alocação financeira para as aquisições}
Considerando o custo total estimado do projeto de R\$ 3.253,87, este será dividido igualitariamente entre os membros da equipe do projeto. Para cada produto que for necessário adquirir será estipulada uma data para compra e o valor será requisitado aos membros da equipe.

\section*{Responsáveis pelas Aquisições}
A responsabilidade pela aquisição será dividida por área, sendo:

\begin{itemize}
	\item Frente de Estrutura - João Kaled\\
	\item Frente Eletroeletrônica - Pedro Henrique Gonçalves Inazawa\\
	\item Frente Eletromecânica - Anderson Andrade Barbosa\\
	\item Frente Interface/Processamento - Matheus Herlan dos Santos Ferraz\\
\end{itemize}

%%%%%%%%%%%%%%%%%%%%%%% FIM PLANO DE GERENCIAMENTO DE AQUISIÇÕES