\chapter[Introdução]{Introdução}\label{cap1}
% 	Essa seção aborda sobre o contexto do projeto, o problema a ser resolvido e a metodologia para desenvolvimento do trabalho.
Vibração é conhecida como o movimento oscilatório de um corpo em relação ao um referencial. O número de ciclos vibratórios do movimento por segundo é chamado de frequência, que é medida em Hertz (Hz) \cite[p. 14]{inman}.

A vibração ocorre em todos os corpos, e testes vibratórios são feitos para determinar as frequências naturais de um determinado corpo, afim de evitar ressonância, e, também, para realizar experimentalmente a durabilidade dinâmica de estruturas permitindo obter o comportamento da estrutura perante vibrações \cite[p. 587]{inman}.

Este trabalho apresenta a proposta de uma bancada para ensaios de vibração mecânica. Assim, neste capítulo são apresentados o problema a ser resolvido, a justificativa e objetivos do projeto.

\section{Descrição do Problema}

    O problema tratado neste projeto pode ser sumarizado de acordo com a Tabela \ref{problem}:

    \begin{table}[h]
        \centering
        \caption{Problema tratado no projeto}
        \label{problem}
        \begin{tabular}{|l|l|}
        \hline
        \textbf{O problema de} & Carência de uma bancada de testes de vibração na FGA. \\ \hline
        \textbf{Afeta} & Os alunos de Engenharia Automotiva e Aeroespacial e  da FGA. \\ \hline
        \textbf{Cujo impacto é} & Formação pouco prática. \\ \hline
        \textbf{Uma boa solução seria} & \begin{tabular}[c]{@{}l@{}}um projeto de uma bancada de testes de vibração genérica\\  para testes de diversas estruturas pelos alunos.\end{tabular} \\ \hline
        \end{tabular}
	\end{table}

\section{Justificativa}
    % Melhorar
    A Faculdade UnB Gama conta com excelentes profissionais e alunos, porém possui uma deficiência na infraestrutura e suporte para os alunos. Alguns equipamentos
    necessários para realização de testes faltam no \textit{campus}, o que
    impacta negativamente a formação dos alunos de Engenharia
    Automotiva e Aeroespacial, principalmente.

    Com isso em mente, este projeto foi proposto para suprir a falta de
    um equipamento de testes de vibração na faculdade, área bastante explorada
    em disciplinas dos cursos de Engenharia Automotiva e Aeroespacial.

\section{Objetivos}

% 	Projetar e construir uma mesa vibratória, de frequência ajustável, que possa ser usada em projetos ou pesquisas, feitas pelas engenharias da UnB-FGA, que necessitem de uma superfície que vibra. Bem como desenvolver sistema de sensoriamento, aquisição e visualização das frequência resposta geradas em teste.

    Este trabalho tem por objetivo geral propor o projeto e implementação de uma bancada para ensaios de vibração mecânica para suprir a carência deste equipamento na UnB-FGA, de forma que possa ser utilizado em projetos e pesquisas feitos pelas engenharias do \textit{campus}.

    % 	A bancada vibratória desse projeto será desenvolvida com o intuito de proporcionar o teste de durabilidade dinâmica de estruturas por meio de vibrações que simulam ambientes críticos, os quais a determinada estrutura terá que enfrentar.

   \subsection{Objetivos específicos}
       São objetivos específicos do projeto:
       \begin{itemize}
%         \item Desenvolver interface visual de baixo nível que permita ao usuário organizar e acompanhar a variação da frequência e tempo durante o uso da mesa construída.
%         \item Projetar e construir sistema eletromecânico que proporcione uma faixa de frequência conhecida ao usuário.
%         \item Criar conjunto de sensores que possam medir o efeito da vibração gerada pela mesa, em pontos diferentes, de objetos.
%         \item Criação de protocolo de comunicação para comunicação entre sensores e micro controladores.
%         \item Escrever manual de uso digital da interface de controle.
%         \item Integrar as soluções de cada frente de trabalho.

        \item Projetar e construir o sistema mecânico/estrutural da bancada;
        \item Projetar e construir o sistema eletromecânico de funcionamento da bancada;
        \item Projetar e construir o sistema eletroeletrônico de funcionamento da bancada;
        \item Projetar e construir o sistema de monitoramento e controle do uso da bancada;
        \item Integrar as soluções de cada frente de trabalho.
       \end{itemize}

       %No Apêndice \ref{enaoe} encontra-se uma lista que identifica características do produto proposto, esclarecendo um pouco mais o escopo do projeto.
	O detalhamento do escopo deste projeto pode ser melhor visualizado no \href{https://drive.google.com/file/d/0B5InkGKx6O-MR1B3eVYzZFpjQ3c/view?usp=sharing}{Relatório 1}. 
       

