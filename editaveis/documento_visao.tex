\chapter{Documento de Visão do Sistema Web}
% 	\input{anexos/Documento_de_visao}

%%%%%%%%%%%%%%%%%%%%%%% DOC DE VISÃO

\begin{center}
 {\large Documento de Visão}\\[0.2cm]
 {BEViM}\\
 \end{center}
 
 \section*{Histórico de Alterações}
\begin{table}[h]
\centering
\begin{tabular}{|c|c|p{6cm}|p{5cm}|}
\hline
Data & Versão & Descrição & Responsável\\
\hline                               
27/08/2016 & 1.0 & Criação do documento. & Ítalo Paiva e Emilie Morais.\\
\hline
\end{tabular}
\end{table}

\section*{Introdução}
	
    O objetivo deste documento é estabelecer uma visão geral do BEViM, que é o sistema que será utilizado para controle e visualização dos resultados da bancada para ensaios de vibração mecânica. Dessa forma, este documento apresenta as necessidades macro do usuário, características gerais do software e envolvidos.
    
   
\section*{Descrições dos Envolvidos e dos Usuários}
	
    Nesta seção do documento são apresentados os envolvidos, os usuários e suas principais necessidades e o ambiente de uso do sistema.
    \subsection*{Resumo dos envolvidos}
		
        Os envolvidos no projeto seja no desenvolvimento, na aquisição ou no uso da aplicação final são \textit{stakeholders}. Foram considerados envolvidos no projeto todos que tenham algum tipo de interesse e/ou participação, e a Tabela \ref{soft_stakeholders} lista os \textit{stakeholders} identificados.

\vfill
\pagebreak
        \begin{table}[h]
            \centering
            \caption{Envolvidos do projeto de Software}
            \label{soft_stakeholders}
            \begin{tabular}{|c|c|c|}
            \hline
            \textbf{Nome}      & \textbf{Descrição}                                                                            & \textbf{Responsabilidades}                                                                                                              \\ \hline
            Emilie Morais      & \begin{tabular}[c]{@{}c@{}}Membro do time\\  de desenvolvimento\end{tabular}                  & Desenvolver e manter o sistema                                                                                                          \\ \hline
            Ítalo Paiva        & \begin{tabular}[c]{@{}c@{}}Membro do time\\  de desenvolvimento\end{tabular}                  & Desenvolver e manter o sistema                                                                                                          \\ \hline
            Matheus Ferraz     & \begin{tabular}[c]{@{}c@{}}Membro do time\\  de desenvolvimento\end{tabular}                  & Desenvolver e manter o sistema                                                                                                          \\ \hline
            Paulo Borba        & \begin{tabular}[c]{@{}c@{}}Membro do time\\  de desenvolvimento\end{tabular}                  & Desenvolver e manter o sistema                                                                                                          \\ \hline
            Demais integrantes & \begin{tabular}[c]{@{}c@{}}Integrantes do grupo de\\  desenvolvimento da bancada\end{tabular} & \begin{tabular}[c]{@{}c@{}}Acompanhar o desenvolvimento \\ e validar a integração do software\\  com os demais subprodutos\end{tabular} \\ \hline
            Professores de PI2 & Professor da disciplina de PI2                                                                & \begin{tabular}[c]{@{}c@{}}Monitorar o andamento do projeto; \\ Avaliar o projeto e o produto.\end{tabular}                             \\ \hline
            \begin{tabular}[c]{@{}c@{}}Alunos \\ (Automotiva e/ou Aeroespacial)\end{tabular} & \begin{tabular}[c]{@{}c@{}}Usuário direto\\  do sistema\end{tabular} & \begin{tabular}[c]{@{}c@{}}Realizar testes no equipamento,\\  controlando a bancada.\end{tabular} \\ \hline
          Técnicos & \begin{tabular}[c]{@{}c@{}}Usuário direto \\ do sistema\end{tabular} & \begin{tabular}[c]{@{}c@{}}Gerenciar e acompanhar \\ o uso do equipamento.\end{tabular} \\ \hline
          \begin{tabular}[c]{@{}c@{}}Professores \\ (Automotiva e/ou Aeroespacial)\end{tabular} & \begin{tabular}[c]{@{}c@{}}Usuário direto\\ do sistema\end{tabular} & \begin{tabular}[c]{@{}c@{}}Realizar testes no equipamento,\\  controlando a bancada\end{tabular} \\ \hline
            \end{tabular}
        \end{table}
        
    \subsection*{Resumo dos usuários}\label{resumo_usuarios_secao}
    
    	Na tabela \ref{usuarios_resumo} estão listados os usuários do sistema.
    
    	\begin{table}[h]
          \centering
          \caption{Resumo dos usuários do sistema}
          \label{usuarios_resumo}
          \begin{tabular}{|c|c|c|}
          \hline
          \textbf{Nome} & \textbf{Descrição} & \textbf{Responsabilidades} \\ \hline
          \begin{tabular}[c]{@{}c@{}}Alunos \\ (Automotiva e/ou Aeroespacial)\end{tabular} & \begin{tabular}[c]{@{}c@{}}Usuário direto\\  do sistema\end{tabular} & \begin{tabular}[c]{@{}c@{}}Realizar testes no equipamento,\\  controlando a bancada.\end{tabular} \\ \hline
          Técnicos & \begin{tabular}[c]{@{}c@{}}Usuário direto \\ do sistema\end{tabular} & \begin{tabular}[c]{@{}c@{}}Gerenciar e acompanhar \\ o uso do equipamento.\end{tabular} \\ \hline
          \begin{tabular}[c]{@{}c@{}}Professores \\ (Automotiva e/ou Aeroespacial)\end{tabular} & \begin{tabular}[c]{@{}c@{}}Usuário direto\\ do sistema\end{tabular} & \begin{tabular}[c]{@{}c@{}}Realizar testes no equipamento,\\  controlando a bancada\end{tabular} \\ \hline
          \end{tabular}
    	\end{table}
    
     \subsection*{Principais necessidades do usuário}
     As necessidades, dos alunos e professores, a serem solucionadas pelo sistema são:
     \begin{itemize}
		\item Iniciar um ensaio na bancada controlando a frequência e o tempo de vibração;
        \item Visualizar os dados de amplitude, frequência, tempo na bancada e no objeto testado;
        \item Obter dados de testes já realizados;
%         \item Agendar horário de uso da bancada;	
    \end{itemize}
    
%     A necessidade, dos técnicos, a ser solucionada pelo sistema é:
%      \begin{itemize}
%         \item Controlar o uso da bancada;
%     \end{itemize}
     
	\subsection*{Ambiente do usuário}
    	
        Os usuários principais são os alunos de Engenharia Automotiva e Aeroespacial e os técnicos, que utilizarão o sistema nos laboratórios da FGA. Como o sistema é uma aplicação \textit{Web}, é possível acessá-lo de qualquer lugar com uma conexão com a Internet.
        
   
\section*{Visão geral do produto}

	\subsection*{Perspectiva e Intenção do produto}
    	Espera-se que o \textit{software} facilite o uso da bancada de ensaios, permitindo um controle mais fácil e uma visualização mais agradável e cômoda dos resultados dos testes, simplificando a vida dos alunos.
        
    \subsection*{Funcionalidades do produto}
    
    	As funcionalidades são um conjunto de características e comportamentos que o sistema deve conter para resolver o problema e as necessidades do cliente. O software BEViM conta com as seguintes funcionalidades:
        
        \begin{itemize}
          	\item \textbf{Controle da bancada} - Esta funcionalidade permite que o usuário controle a entrada de dados para realizar um ensaio de vibração na bancada.
            \item \textbf{Visualização dos resultados} - Esta funcionalidade permite que o usuário visualize os resultados dos testes realizados, por meio de gráficos amigáveis.
        \end{itemize}
    
    \subsection*{Requisitos não-funcionais do produto}
    
    	Requisitos não-funcionais são regras que, mesmo não sendo funcionalidades, o sistema deve estar de acordo. Para o BeViM os seguintes requisitos não-funcionais foram identificados:
        
        \begin{itemize}
            \item Deve ser possível apenas a execução de um teste por vez, por mesa.
            \item Os resultados do teste devem ser apresentados de forma amigável, com gráficos inteligíveis para os alunos de Engenharia.
            \item O sistema deve possuir um sistema de autenticação para a realização do teste.

        \end{itemize}
        
        \subsubsection*{Restrições de Design}
        	
            \begin{itemize}
                \item O sistema será desenvolvido na linguagem de programação Python 3.4, utilizando o \textit{framework} de desenvolvimento web Django versão 1.9.
                \item O sistema deverá ser implementado seguindo a folha de estilos padrão do Python, o \href{https://www.python.org/dev/peps/pep-0008/}{PEP 8}, prezando a manutenibilidade, para que o código possa ser facilmente evoluído por outros alunos.
                \item O desenvolvimento do sistema deve seguir a metodologia ágil \textit{Scrum}.
                
                \item O versionamento do código deverá ser feito via GitHub.
            \end{itemize}
    
\section*{Glossário}
    
    \textbf{Ensaio:} Teste realizado pela bancada 
 
    \textbf{Bancada:} Estrutura vibratória criada para a realização dos ensaios 
    
    \textbf{Vibração mecânica:} Vibração é conhecida como o movimento oscilatório de um corpo em relação ao um referencial. O número de ciclos vibratórios do movimento por segundo é chamado de frequência que é medido em Hertz (Hz) \cite{inman}.
    
%%%%%%%%%%%%%%%%%%%%%%% FIM DO DOC DE VISÃO