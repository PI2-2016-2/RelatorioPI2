\chapter{Plano de Trabalho - Projeto de Melhoria de Processos - Equipe SiGA}
\label{plano_de_projeto}
\section*{Escopo}

\subsection*{Objetivos}

\textbf{Necessidades do negócio:} 
Melhorar a qualidade dos produtos entregues.

\textbf{Motivações para melhoria:} 
\begin{itemize}
	\item Baixa qualidade dos produtos entregues;
	\item Dificuldades na implantação;
	\item Despadronização das formas de trabalho;
	\item Falta de monitoramento do projeto.
\end{itemize}

\textbf{Objetivos de melhoria:} 
\begin{itemize}
	\item Formalizar o processo de desenvolvimento de software;
	\item Otimizar a atividade de implantação;
	\item Estabelecer métricas para o processo.
\end{itemize}

\subsection*{Produtos Relevantes}

	Os produtos relevantes para o projeto de melhoria de processos da equipe SiGA são:
	
	\begin{itemize}
		\item Plano de Projeto de MPS;
		\item O processo de desenvolvimento de software;
		\item Cronograma;
		\item Estrutura analítica de risco;
		\item Relatório de Acompanhamento do projeto de MPS;
		\item Lista de melhorias.
	\end{itemize}


\section*{Recursos}

\subsection*{Recursos Humanos}

	Os recursos humanos alocados para o projeto são os dois desenvolvedores com dedicação total. Para orientação do projeto será alocado um orientador com disponibilidade para o projeto de 2 vezes por semana.


\subsection*{Recursos Materiais e de Infraestrutura}

	A equipe já possui os recursos materiais e de infraestrutura necessários para execução do projeto.

\section*{Cronograma}

\begin{figure}[!ht]
\centering
\includegraphics[scale=0.8]{figuras/cronograma.png}
\caption{Cronograma do projeto}
\label{fig:cronograma}
\end{figure}

\section*{Riscos}
  
  \input{editaveis/riscos}

\section*{Plano de Comunicações Relevantes}

	\begin{itemize}
		\item Aprovação do processo definido;
		\item Acompanhamento do projeto;
		\item Divulgação dos resultados.

	\end{itemize}

\section*{Plano de Monitoramento}
	
	O monitoramento do projeto de MPS será realizado semanalmente juntamente com o orientador.

