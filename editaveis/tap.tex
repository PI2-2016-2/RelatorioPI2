\chapter{Termo de Abertura do Projeto}
	\label{tap}
% 	\input{anexos/Termo_de_abertura}

%%%%%%%%%%%%%%%%%%%%%%%% INICIO DO TAP
\begin{center}
 {\large Termo de abertura}\\[0.2cm]
 {Bancada para Ensaios de Vibração Mecânica}\\
 \end{center}
 
 \section*{Histórico de Alterações}
\begin{table}[h]
\centering
\begin{tabular}{|c|c|p{6cm}|p{5cm}|}
\hline
Data & Versão & Descrição & Responsável\\
\hline                               
26/08/2016 & 1.0 & Criação do documento. & Anderson Tenório, Ítalo Paiva e Paulo Borba .\\
\hline                               
27/08/2016 & 1.1 & Reformulação da estrutura do documento. & Ítalo Paiva.\\
\hline
\end{tabular}
\end{table}

\section*{Nome do projeto}
  Bancada para Ensaios de Vibração Mecânica.
  
\section*{Descrição do projeto}

    O projeto visa desenvolver uma bancada para ensaios de vibração mecânica
    genérica para os alunos dos cursos de Engenharia Automotiva e Aeroespacial
    da UnB-FGA (Faculdade do Gama) possam realizar testes de diversas estruturas
    nas disciplinas que os fizerem necessários.

\section*{Objetivos do projeto}
  
    Este trabalho tem por objetivo geral propor o projeto e implementação
    de uma bancada para ensaios de vibração mecânica para suprir a carência
    deste equipamento na UnB-FGA, que possui uma aplicação mais didática
    que industrial.

   São objetivos específicos do projeto:
   \begin{itemize}
    \item Elaborar o projeto mecânico/estrutural da bancada;
    \item Elaborar o projeto eletromecânico de funcionamento da bancada;
    \item Elaborar o projeto eletroeletrônico de funcionamento da bancada;
    \item Elaborar o projeto do sistema de monitoramento e controle da bancada;
    \item Integrar as soluções de cada frente de trabalho.
   \end{itemize}
  
\section*{Justificativa do projeto}
	
    %%%%%%%%%%%%%% Melhorar isso aqui
    
    A Faculdade UnB Gama conta com excelentes profissionais e alunos, porém 
    a falta de equipamentos especializados e infraestrutura para que os alunos 
    possam aprender melhor, é um problema antigo. Alguns equipamentos
    necessários para realização de testes faltam no \textit{campus}, o que
    acaba impactando negativamente a formação dos alunos de Engenharia
    Automotiva e Aeroespacial, principalmente.
    
    Com isso em mente, este projeto foi proposto para suprir a falta de
    um equipamento de testes de vibração na faculdade, área bastante explorada
    em disciplinas dos cursos de Engenharia Automotiva e Aeroespacial.
    
    %%%%%%%%%%%%%% Melhorar isso aqui

\section*{Recursos do produto}
	
    A solução apresentada conta com os seguintes recursos:
    
    \begin{itemize}
        \item Frequência controlada pelo usuário;
    	\item Calibração automática da frequência da mesa;
        \item Sensoreamento livre em dois locais de escolha do usuário;
        \item Aplicação \textit{web} para controle da bancada e visualização
        	  dos resultados.
    \end{itemize}
    
    \subsection*{Subprodutos identificados}
    
    	O projeto como um todo gerará os seguintes subprodutos:
        
        \begin{itemize}
        	\item \textbf{Estrutura física da bancada vibracional} - Projeto e produto
            			  da estrutura de sustentação, do mecanismo vibracional e da
                          estrutura da bancada. A bancada vibracional engloba tanto a
                          mesa vibratória e sua sustentação quanto os 
                          sensores acoplados na mesa para controle da frequência e os
                          sensores livres para o usuário.
        	\item \textbf{Sistema de Monitoramento e Controle da Bancada} - 
                          Aplicação \textit{web} responsável por controlar o
                          funcionamento da bancada, onde o usuário entrará com
                          os dados de entrada para o teste e visualizará os
                          resultados do teste. Além da aplicação, será fornecida
                          também toda a documentação associada.
        \end{itemize}

\section*{\textit{Stakeholders}}

Os envolvidos no projeto seja no desenvolvimento, na aquisição ou no uso da aplicação final são \textit{stakeholders}. Foram considerados envolvidos no projeto todos que tenham algum tipo de interesse e/ou participação, e a Tabela \ref{stakeholders_projeto} lista os \textit{stakeholders} identificados.
        
        \begin{table}[h]
            \centering
            \caption{Envolvidos no projeto}
            \label{stakeholders_projeto}
            \begin{tabular}{|c|c|c|}
            \hline
            \textbf{Nome}      & \textbf{Descrição}                                                                            & \textbf{Responsabilidades}                                                                                                              \\ \hline
            Integrantes do grupo & \begin{tabular}[c]{@{}c@{}}Integrantes do grupo de\\  desenvolvimento da bancada\end{tabular} & \begin{tabular}[c]{@{}c@{}}Acompanhar o desenvolvimento \\ e validar a integração do software\\  com os demais subprodutos\end{tabular} \\ \hline
            Professores de PI2 & Professor da disciplina de PI2                                                                & \begin{tabular}[c]{@{}c@{}}Monitorar o andamento do projeto; \\ Avaliar o projeto e o produto.\end{tabular}                             \\ \hline
            \begin{tabular}[c]{@{}c@{}}Alunos \\ (Automotiva e/ou \\ Aeroespacial)\end{tabular} & \begin{tabular}[c]{@{}c@{}}Usuário direto\\  do sistema\end{tabular} & \begin{tabular}[c]{@{}c@{}}Realizar testes no equipamento,\\  controlando a bancada.\end{tabular} \\ \hline
          Técnicos & \begin{tabular}[c]{@{}c@{}}Usuário direto \\ do sistema\end{tabular} & \begin{tabular}[c]{@{}c@{}}Gerenciar e acompanhar \\ o uso do equipamento.\end{tabular} \\ \hline
          \begin{tabular}[c]{@{}c@{}}Professores \\ (Automotiva e/ou \\ Aeroespacial)\end{tabular} & \begin{tabular}[c]{@{}c@{}}Usuário direto\\ do sistema\end{tabular} & \begin{tabular}[c]{@{}c@{}}Realizar testes no equipamento,\\  controlando a bancada\end{tabular} \\ \hline
            \end{tabular}
        \end{table}


\section*{Líderes do projeto e suas responsabilidades}

    Cada frente de trabalho possui uma interface de comunicação com o grupo todo.
    Esta interface é o líder da frente. 

    Então, o líder fica responsável por coordenar e dividir atividades entre a sua
    frente de trabalho e comunicar à equipe decisões, status de atividades e entre
    outras informações. Os líderes de cada frente, em conjunto com o grupo, são
    responsáveis por tomar decisões e representar o grupo como um todo.

    A relação dos líderes de cada frente de trabalho definida se encontra abaixo:

    \begin{itemize}
        \item \textbf{Frente Estrutural/Mecânica} - João Kaled
        \item \textbf{Frente Eletromecânica} - Anderson Andrade
        \item \textbf{Frente Eletroeletrônica} - Pedro Inazawa
        \item \textbf{Frente de Interface/Processamento} - Matheus Ferraz
    \end{itemize}

\section*{Cronograma de marcos sumarizado}

	A Tabela \label{tab:cronograma_marcos} apresenta os principais marcos do projeto em que serão responsáveis para realização
    de avaliações sobre o estado do projeto bem como entregas de protótipos funcionais e produto final,
    com as devidas datas de acordo com o cronograma.

  \begin{table}[h]
  \centering
  \label{tab:cronograma_marcos}
  \begin{tabular}{|c|p{4.5cm}|p{8cm}|}
	\hline
  Data & Marcos & Descrição\\
  \hline                               
  19/08-06/08/2016 & Solução inicial & Definição da solução inicial do projeto a partir de ideias integradoras.\\
  \hline                               
  09/09/2016 & Ponto de Controle 1 & Apresentação da concepção do projeto com os requisitos bem definidos 
  									sobre o que será entregue e quais as características serão 												desenvolvidas para o produto a ser entregue.\\
  \hline                               
  12/09-03/11/2016 & Construção da solução & Desenvolvimento físico da bancada para ensaios de vibração mecânica.\\
  \hline                               
  09/11/2016 & Ponto de Controle 2 & Apresentação de um protótipo funcional e relatório do 
  									andamento do projeto.\\
  \hline                               
  26/10-30/11/2016 & Integração e Homologação & Integração das soluções e homologação da bancada para ensaios de vibração mecânica após realizados todos os testes.\\
  \hline
  02/12/2016 & Ponto de Controle 3 e Entrega Final do Produto & Apresentação do produto final, com os módulos 																	de cada engenharia integrados e que 																	estejam de acordo com os requisitos 																	elicitados inicialmente.\\
  \hline
  \end{tabular}
  \end{table}


\section*{Investimento preliminar}

	O custo total estimado do projeto é de R\$ 3.253,87.

\section*{Restrições e riscos}
	
    Os riscos identificados para o projeto estão listados na Tabela \ref{riscos_projeto_tap}, juntamente com a probabilidade, impacto e prioridade definidos (mais informações sobre os riscos do projeto podem ser obtidos no Plano de Gerenciamento de Riscos, no Apêndice \ref{plano_de_riscos}).

\begin{table}[h]
\centering
\caption{Riscos Identificados para o Projeto}
\label{riscos_projeto_tap}
\begin{tabular}{|l|l|l|c|}
\hline
\multicolumn{1}{|c|}{\textbf{Risco}} & \multicolumn{1}{c|}{\textbf{Probabilidade}} & \multicolumn{1}{c|}{\textbf{Impacto}} & \textbf{Prioridade} \\ \hline
Algum membro abandonar a equipe & Baixa & Alto & \cellcolor[HTML]{FCFF2F}\textbf{Média} \\ \hline
Não houver financiamento suficiente. & Média & Alto & \cellcolor[HTML]{FE0000}\textbf{Alta} \\ \hline
\begin{tabular}[c]{@{}l@{}}Não conseguir os materiais\\  necessários para o produto por\\  tempo e localização.\end{tabular} & Baixa & Alto & \cellcolor[HTML]{FCFF2F}\textbf{Média} \\ \hline
Houver atrasos entre as equipes. & Média & Alto & \cellcolor[HTML]{FE0000}\textbf{Alta} \\ \hline
\begin{tabular}[c]{@{}l@{}}Os horários do galpão não serem\\  suficientes ou aptos.\end{tabular} & Alta & Médio & \cellcolor[HTML]{FE0000}\textbf{Alta} \\ \hline
\begin{tabular}[c]{@{}l@{}}Requisito entre as equipes serem\\ modificados no decorrer do processo.\end{tabular} & Média & Alto & \cellcolor[HTML]{FE0000}\textbf{Alta} \\ \hline
Greve & Baixa & Baixo & \cellcolor[HTML]{34FF34}\textbf{Baixa} \\ \hline
\end{tabular}
\end{table}

%%%%%%%%%%%%%%%%%%%%%%%% FIM DO TAP