\begin{resumo}
% O presente trabalho analisa, caracteriza, viabiliza e perscruta  o projeto de construção de uma bancada para ensaio de vibração mecânica, 
% na qual parte-se do princípio físico que as vibrações podem excitar as frequências naturais das peças que compõem o sistema e com isso,
% pode-se conseguir uma avaliação exata dessa vibração por meio de medição e análise, sendo utilizado acelerômetros 
% piezoelétricos coma a finalidade de converter o movimento vibratório em sinais elétricos. Assim, Empregando-se a técnica de 
% análise de frequência com as entradas fornecidas no software pelo usuário processara-se um espectrograma de frequência, ou seja, 
% um histograma que relaciona a amplitude ou nível do sinal com a sua respectiva frequência, onde a amplitude será 
% quantificada de diversas maneiras, tais como: nível pico-pico, nível pico, nível médio e o nível quadrático médio ou valor eficaz (RMS).

Vibração mecânica é caracterizada pelo movimento oscilatório de um corpo em relação a um referencial.
Estes movimentos podem excitar as frequências naturais das peças que compõem um sistema. Desse modo, testes vibratórios
são realizados para que o comportamento de uma determinada estrutura perante vibrações seja obtido. Considerando este cenário, este 
trabalho apresenta a proposta e execução de um projeto de construção de uma bancada para ensaio de vibração mecânica que contemple
a integração de diferentes áreas da engenharia. A bancada apoia a análise do comportamento da estrutura
a partir de frequências fornecidas pelo usuário por meio de um software que estabelece a comunicação com um sistema de
controle eletroeletrônico e em seguida eletromecânico. O sistema de controle aciona o motor que faz com que estrutura vibre na
frequência estabelecida.

\vspace{\onelineskip}
    
 \noindent
 \textbf{Palavras-chave}: vibração,frequência,bancada,medição, análise
 
\end{resumo}
