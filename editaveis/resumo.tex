\begin{resumo}
O presente trabalho analisa, caracteriza, viabiliza e perscruta  o projeto de construção de uma bancada para ensaio de vibração mecânica, na qual parte-se do princípio físico que as vibrações podem excitar as frequências naturais das peças que compõem o sistema e com isso, pode-se conseguir uma avaliação exata dessa vibração por meio de medição e análise, sendo utilizado acelerômetros piezoelétricos coma a finalidade de converter o movimento vibratório em sinais elétricos. Assim, Empregando-se a técnica de análise de frequência com as entradas fornecidas no software pelo usuário processara-se um espectrograma de frequência, ou seja, um histograma que relaciona a amplitude ou nível do sinal com a sua respectiva frequência, onde a amplitude será quantificada de diversas maneiras, tais como: nível pico-pico, nível pico, nível médio e o nível quadrático médio ou valor eficaz (RMS).
  


 \vspace{\onelineskip}
    
 \noindent
 \textbf{Palavras-chave}: vibração,frequência,bancada,medição, análise
 
\end{resumo}
