\chapter{Considerações finais}

% Ao final das semanas de trabalho para a entrega do primeiro ponto de controle foi possível analisar de forma mais 
% concisa os aspectos técnicos inerentes ao projeto da bancada de vibrações. Contudo, vale ressaltar que algumas decisões ainda 
% serão efetuadas com o propósito de melhorar as soluções providas pelas frentes de trabalho.

% De acordo com a análise de viabilidade e também com os demais quesitos da gerência do projeto, constatou-se que o mesmo é viável, 
% tanto da perspectiva de custos como também da de prazos.

% Como trabalhos futuros, tem-se para o segundo ponto de controle a apresentação funcional dos subsistemas estabelecidos pelas frentes 
% de trabalho e, ao final, no terceiro ponto de controle, será apresentado o produto  completo, com todos os subsistemas integrados.

Nas fases de Concepção e Detalhamento da Solução foi possível analisar e definir de forma concisa os aspectos técnicos inerentes 
ao projeto. Além disso, o projeto foi considerado viável a partir da realização de uma 
análise de viabilidade de acordo com aspectos relacionados aos custos, prazos e desafios técnicos.

No término das duas primeiras fases foram percebidos problemas na gestão da escrita do relatório e no gerenciamento de atividades. Desse modo, a ferramenta para gerenciamento do relatório foi trocada e para o gerenciamento das atividades foi definido que não seria mais utilizada uma ferramenta, pois foram percebidos problemas de atualização. Além disso, o problema a ser resolvido foi melhor compreendido o que acarretou 
em algumas mudanças na solução definida.

Na fase de Projeto e Construção a solução definida para cada frente foi refinada. A partir deste refinamento 
realizado os subsistemas provenientes de cada frente foram projetados e construídos. 
A integração entre as frentes foi estabelecida para que os projetos de cada subsistema 
fossem feitos de forma a possibilitar a integração da fase seguinte. No final desta fase para cada subsistema foram realizados
testes a fim de verificar e validar a construção realizada. 

Alguns fatores impactaram no planejamento inicial, todavia, as atividades planejadas para o projeto até a fase relatada neste trabalho foram realizadas. 
Para o próximo relatório será apresentada a fase de Integração cujo objetivo é formar a solução proposta neste trabalho a partir da integração
dos subsistemas.

% 
