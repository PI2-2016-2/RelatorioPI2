\chapter{Considerações finais}

% Ao final das semanas de trabalho para a entrega do primeiro ponto de controle foi possível analisar de forma mais
% concisa os aspectos técnicos inerentes ao projeto da bancada de vibrações. Contudo, vale ressaltar que algumas decisões ainda
% serão efetuadas com o propósito de melhorar as soluções providas pelas frentes de trabalho.

% De acordo com a análise de viabilidade e também com os demais quesitos da gerência do projeto, constatou-se que o mesmo é viável,
% tanto da perspectiva de custos como também da de prazos.

% Como trabalhos futuros, tem-se para o segundo ponto de controle a apresentação funcional dos subsistemas estabelecidos pelas frentes
% de trabalho e, ao final, no terceiro ponto de controle, será apresentado o produto  completo, com todos os subsistemas integrados.

%%%%%%%%%%%%%%%%% REFAZER %%%%%%%%%%%%%

% Nas fases de Concepção e Detalhamento da Solução foi possível analisar e definir de forma concisa os aspectos técnicos inerentes
% ao projeto. Além disso, o projeto foi considerado viável a partir da realização de uma
% análise de viabilidade de acordo com aspectos relacionados aos custos, prazos e desafios técnicos.

% No término das duas primeiras fases foram percebidos problemas na gestão da escrita do relatório e no gerenciamento de atividades. Desse modo, a ferramenta para gerenciamento do relatório foi trocada e para o gerenciamento das atividades foi definido que não seria mais utilizada uma ferramenta, pois foram percebidos problemas de atualização. Além disso, o problema a ser resolvido foi melhor compreendido o que acarretou
% em algumas mudanças na solução definida.

% Na fase de Projeto e Construção a solução definida para cada frente foi refinada. A partir deste refinamento
% realizado os subsistemas provenientes de cada frente foram projetados e construídos.
% A integração entre as frentes foi estabelecida para que os projetos de cada subsistema
% fossem feitos de forma a possibilitar a integração da fase seguinte. No final desta fase para cada subsistema foram realizados
% testes a fim de verificar e validar a construção realizada.

% Alguns fatores impactaram no planejamento inicial, todavia, as atividades planejadas para o projeto até a fase relatada neste trabalho foram realizadas.
% Para o próximo relatório será apresentada a fase de Integração cujo objetivo é formar a solução proposta neste trabalho a partir da integração
% dos subsistemas.

As considerações finais deste projeto é dividido em duas partes, uma sobre as conclusões técnicas, bem como dificuldades técnicas em surgir com a solução de determinado problema, e por fim a lição aprendida por todas as frentes e a dificuldade de integrar em um curto espaço de tempo.

A Bancada para ensaios de vibração mecânica possui 4 frentes de integração, sendo Estrutura com Eletromecânica (responsáveis pelo motor girar o eixo desbalanceado), Eletromecânica com Eletroeletrônica (responsáveis por controlar o motor através de circuitos), Eletroeletrônica com Estrutura (monitoramento dos sensores que estão no tampo e no objeto de teste da bancada) e por fim Eletroeletrônica com Interface \& Processamento.

Produto final saiu quase completo, mas por detalhes de atraso no cronograma, fez com que atrasassemos na entrega do produto final e por fim a integração final que acabou ficando realizada parcialmente, pois os testes de validação do sistema integrado não foram possíveis de serem realizados para garantir a integridade do produto. Mas apesar deste fato, todas as integrações foram realizadas, e a frente de Eletrônica foi a que mais teve dificuldade pois é a frente núcleo de todo o sistema, pois ela se integra com as 3 frentes ao mesmo tempo.

Falta de um gerenciamento mais duro e planejamento adequado, foram motivos que levaram a este fator acontecer mas mesmo apesar destes fatores, ainda foi possível surgir com o produto final que segue a risca os requisitos que foram determinados nas primeiras fases.

A conclusão que se leva ao fim de um projeto em grande escala em um curto espaço de tempo é que todo o tempo é precioso e precisar ser bem mensurado, não pode-se ter pessoas ociosas no grupo, pois isso significa que atividades que precisam ser feitas irão se atrasar.


%%%%%%%%%%%%% TODO -  Lições aprendida %%%%%%%%%%%
\section*{Lições aprendidas}

Para as lições aprendidas neste último relatório é possível dizer que o maior problema de um projeto multidisciplinar, é a disciplina e organização dos membros. Como todo projeto, existe a necessidade de se aplicar técnicas de planejamento, organização, comunicação para que se possa ter um controle das atividades e do andamento do projeto.

A disciplina de Projeto Integrador 2, vai além de uma mera disciplina de projeto, ela vêm com uma importante lição que é saber lidar com pessoas e prazos. Uma simulação breve sobre como o mercado de trabalho funciona. Essa matéria tem como objetivo mostrar na prática desde a concepção de um produto até a sua construção dos módulos e por fim a integração deles para disponibilizar no mercado.

Um dos maiores problemas que o grupo enfrentou foi a falta de organização por parte dos sub-grupos mesmo tendo uma organização pré-definida com prazos para entrega, mas a falta de qualidade nas entregas e as vezes o atraso leva com que os membros do grupo não levem a sério a disciplina. E a principal mensagem que a disciplina quer transmitir é saber se os alunos estão prontos para lidar com a rigidez do mercado de trabalho.

\subsection*{Engenharia Automotiva e Aeroespacial}

\subsection*{Engenharia de Energia}

\subsection*{Engenharia de Eletrônica}

A equipe de eletrônica teve alguns desafios tanto do ponto de vista técnico quanto no de organização da equipe, que precisaram ser superados para que o trabalho final fosse viabilizado.

O primeiro desafio do projeto foi coordenar e nivelar os conhecimentos da equipe de eletronica com os das demais áreas. Como esse projeto possui muitos pontos relacionados a análise de estrutura, fadiga, e outros, a equipe teve que se adaptar ao vocabulário da equipe de automotiva e de aeroespacial, para que por fim pudesse entender os testes a serem feitos, e propor soluções.

Logo em seguida, vieram os desafios técnicos de se integrar às soluções propostas pelos outros times. A equipe de eletrônica atuou como o centro do projeto, conectando o alto nível da aplicação (o software) com o baixo nível (o inversor), tudo isso atendendo a requisitos de temporização e controle previstos pelos demais times.

Por fim, o principal desafio foi realmente trabalhar e dividir tarefas para que tudo saísse confome planejado. Dentro da equipe, 2 pessoas já trabalhavam em empresas externas a faculdade, e uma estava fazendo TCC, o que fez do tempo um recurso escasso. Mesmo assim, a equipe trabalhou de forma dinâmica, tentando ao máximo que o resultado final fosse alcançado.

Concluíndo, a disciplina de PI2 foi cheia de lições valorosas para a equipe de eletronica, tanto do ponto de vista de construção profissional, quanto de aprendizado técnico sobre engenharia. Acreditamos que essa disciplina foi um diferencial para nós, e que agora  estamos um pouco mais próximos da realidade do mercado de trabalho.

\subsection*{Engenharia de Software}
%
A equipe de software conta com pessoas responsáveis, e que sempre procuraram ajudar não somente entre os próprios membros de software, mas sim o grupo como um todo. Pode-se dizer que software agiu parcialmente como gerentes do grupo tentando organizar, criando prazos e gerenciando da melhor maneira possível. Além de terem que fazer o trabalho da frente a qual pertencem, que não foi um trabalho muito fácil, até por que Software agiu tanto no baixo nível quanto no alto nível com relação a nível de aplicação.

A maior dificuldade que software teve foi tentar organizar o grupo como um todo e tentar organizar as integrações de forma que elas ocorressem no tempo determinado, mas a falta de disciplina e organização por parte de alguns membros fez com que isso não saísse como o esperado. Atrasando a entrega do produto final e diminuindo sua qualidade.

Software, teve seus desafios exclusivos principalmente a manipulação a nível de bits e bytes com eletrônica, para garantir a integração com eletrônica, mas com o atraso do grupo de Eletrônica, essa integração do produto como um todo demorou para sair. O que se aprende em um projeto desse na parte de Software, é que os requisitos podem ser feitos e realizados de diversas maneiras, e eles sempre estarão passivos de mudança ao longo da evolução do projeto, para considerar fatores que não foram considerados. Ao fim deste projeto, software entregrou seu módulo integrado exclusivamente com eletrônica, e ainda ajudando ao máximo o grupo no que desse para garantir a integração final do produto, mesmo que atrasada.