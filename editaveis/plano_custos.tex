

\chapter{Plano de Gerenciamento de Custos}
	\label{plano_de_custos}
% 	\input{anexos/Plano_de_Gerenciamento_de_Custos}
    
%%%%%%%%%%%%%%%%%%%%%%% PLANO DE GERENCIAMENTO DE CUSTOS

\begin{center}
 {\large Plano de Gerenciamento de Custos}\\[0.2cm]
 {Bancada para Ensaios de Vibração Mecânica}\\
 \end{center}
 
 \section*{Histórico de Alterações}
\begin{table}[h]
\centering
\begin{tabular}{|c|c|p{6cm}|p{5cm}|}

Data & Versão & Descrição & Responsável\\
\hline                               
01/09/2016 & 1.0 & Criação deste documento & Emilie Morais\\ \hline
\end{tabular}
\end{table}

\section*{Objetivo}
  O objetivo desse plano é definir como será realizado o gerenciamento de custos do projeto, bem como apresentar o custo total estimado.

  
\section*{Descrição dos processos de gerenciamento de custo}

Primeiramente, será realizada a estimativa do custo de aquisições do projeto, pois dado o contexto do projeto não será calculado o custo de mão-de-obra. A partir do custo estimado será definido o orçamento do projeto, considerando cada integrante.

Durante a execução do projeto os custos serão monitorados a fim de visualizar o custo planejado e o custo real.

% Considerando a solução proposta no projeto foram estabelecidos os materiais que serão utilizados para a construção da bancada, considerando também os componentes eletrônicos e serviço de hospedagem. 
% Para cada material foi pesquisado um valor e dessa forma o custo total do projeto foi estimado. 

% para ter base da viabilidade do projeto e que seu custo não ultrapasse o estabelecido no orçamento do projeto.

\section*{Alocação financeira das mudanças no orçamento}

As mudanças serão comunicadas ao grupo que decidirá sobre o orçamento disponível.

%%%%%%%%%%%%%%%%%%%%%%% FIM PLANO DE GERENCIAMENTO DE CUSTOS