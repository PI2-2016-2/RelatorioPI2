\chapter{Documento de Especificação dos Casos de Teste} \label{documento_teste}

\begin{center}
 {\large Documento de Especificação dos Casos de Teste}\\[0.2cm]
 \end{center}

 \section*{Histórico de Alterações}
\begin{table}[h]
\centering
\begin{tabular}{|c|c|p{6cm}|p{5cm}|}
\hline
Data & Versão & Descrição & Responsável\\
\hline
01/11/2016 & 1.0 & Criação do documento. & Paulo Borba e Emilie Morais.\\
\hline
\end{tabular}
\end{table}

\section*{Introdução}

Este documento apresenta os casos de teste a serem realizados nos subsistemas da bancada com o intuito de validar os requisitos
definidos e garantir a integridade do sistema como um todo. As tabelas abaixo trazem o detalhamento dos casos de testes contendo os tipos de testes que serão 
realizados, os subsistemas participantes e os resultados esperados.
\vfill
\begin{table}[]
    \begin{center}
        \begin{longtable}{|p{5cm}|p{12cm}|}
            \hline
            \multicolumn{2}{|c|}{\textbf{Especificação de Caso de Teste}} \\ \hline
                \textbf{Subsistemas}                               & Interface/Processamento \\ \hline
                \textbf{Caso de Teste}                             & CT-01 \\ \hline
                \textbf{Propósito}                                     & Realizar o login do usuário \\ \hline
                \textbf{Descrição Geral}                           & Este teste valida como o sistema se comporta caso o usuário e a senha sejam válidos ou inválidos. O sistema foi projetado para realizar um tratamento de usuário e senha inválidos. \\ \hline
                \textbf{Insumos para o caso de Teste}    & O servidor Django tem que estar ativado. \\ \hline
                \textbf{Roteiro para realização do Teste}&  1. Realizar as migrações dos dados da model para o Banco de Dados local, 2. Criar um usuário de teste via terminal do banco de dados, 3. Levantar o servidor Django localmente, 4. Logar na página da aplicação com o usuário criado. \\ \hline
            \multicolumn{2}{|c|}{\textbf{Cenários de Teste}} \\ \hline
                \textbf{Objetivo Específico}                      & \textbf{Saídas Esperadas} \\ \hline
		  Tratar o login de usuário válido                           & Sistema deverá redirecionar para a página inicial.                                                                                                                                                                                    \\ \hline
		  Tratar o login de usuário que não existe na base de dados. & Sistema deverá exibir uma exceção dizendo que o usuário não existe.                                                                                                                                                                   \\ \hline
		  Tratar a senha errada                                      & Sistema deverá informar que a senha está errada e pedir para digite novamente.   
        \end{longtable}
    \end{center}
    \caption[Caso de Teste - CT01]{Caso de Teste - CT01
    \protect Fonte: Autor}
    \label{CT-01}
\end{table}

\begin{table}[H]
    \begin{center}
        \begin{tabular}{|p{5cm}|p{12cm}|}
            \hline
            \multicolumn{2}{|c|}{\textbf{Especificação de Caso de Teste}} \\ \hline
                \textbf{Subsistemas}                               & Interface/Processamento \\ \hline
                \textbf{Caso de Teste}                             & CT-02 \\ \hline
                \textbf{Propósito}                                     & Comunicação do servidor com a aplicação \\ \hline
                \textbf{Descrição Geral}                           & Este teste valida como o sistema deverá se comportar no envio de requisições. \\ \hline
                \textbf{Insumos para o caso de Teste}    & O servidor Django tem que estar ativado. \\ \hline
                \textbf{Roteiro para realização do Teste}&  1. Subir o servidor Django, 2. Realizar uma requisição GET com um resultado cujo seja esperado e comparar se o resultado obtido é o mesmo do esperado, 3. Realizar uma requisição POST em uma página de forma a saber se o POST está funcionando devidamente. \\ \hline
            \multicolumn{2}{|c|}{\textbf{Cenários de Teste}} \\ \hline
                \textbf{Objetivo Específico}                      & \textbf{Saídas Esperadas} \\ \hline
                Realizar uma requisição para o REST & Sistema apresentar a página referente \\ \hline
        \end{tabular}
    \end{center}
    \caption[Caso de Teste - CT02]{Caso de Teste - CT02
    \protect Fonte: Autor}
    \label{CT-02}
\end{table}

\begin{table}[H]
    \begin{center}
        \begin{tabular}{|p{5cm}|p{12cm}|}
            \hline
            \multicolumn{2}{|c|}{\textbf{Especificação de Caso de Teste}} \\ \hline
                \textbf{Subsistemas}                               & Interface/Processamento\\ \hline
                \textbf{Caso de Teste}                             & CT-03 \\ \hline
                \textbf{Propósito}                                     & Testar conexão, envio e leitura de dados com o Banco de Dados Django \\ \hline
                \textbf{Descrição Geral}                           & Este teste tem como objetvido validar a conexão com o banco de dados da Raspberry, bem como escrita e leitura desses dados. O sistema deve estar preparado para tratar caso a conexão com o banco falhe ou tenha um estouro de memória devido a quantidade de informações que estejam transitando esteja muito alta. \\ \hline
                \textbf{Insumos para o caso de Teste}    & O servidor Django Raspberry tem que estar ativo. \\ \hline
                \textbf{Roteiro para realização do Teste}&  1. Subir o servidor Django da raspberry, 2.Utilizando o Parser, realizar a conexão com o banco de dados local, 3. Realizar a inserção de um dado em uma tabela, 4. Realizar a leitura deste mesmo dado na tabela, 5. Averiguar se o dado guardado é o mesmo lido  \\ \hline
            \multicolumn{2}{|c|}{\textbf{Cenários de Teste}} \\ \hline
                \textbf{Objetivo Específico}                      & \textbf{Saídas Esperadas} \\ \hline
                Tratar caso a conexão com o banco de dados não seja estabelecida & Sistema deverá informar que o banco de dados local está corrompido devido algum problema de hardware/software que venha a ter prejudicado a integridade dos dados \\ \hline
                Tratar caso aconteça estouro de memória para evitar que o sistema trave devido a quantidade de informações transitando & Sistema deverá informar o percentual de uso da memória interna, do disco rígido e de processamento, e quando as 3 variáveis chegarem a um limite o sistema deverá lançar um anuncio ao usuário informando o risco de possível travamento do sistema.  \\ \hline
        \end{tabular}
    \end{center}
    \caption[Caso de Teste - CT03]{Caso de Teste - CT03
    \protect Fonte: Autor}
    \label{CT-03}
\end{table}

\begin{table}[H]
    \begin{center}
        \begin{tabular}{|p{5cm}|p{12cm}|}
            \hline
            \multicolumn{2}{|c|}{\textbf{Especificação de Caso de Teste}} \\ \hline
                \textbf{Subsistemas}                               & Interface/Processamento\\ \hline
                \textbf{Caso de Teste}                             & CT-04 \\ \hline
                \textbf{Propósito}                                     & Testar conexão, envio e leitura de dados com o Banco de Dados Django \\ \hline
                \textbf{Descrição Geral}                           & Este teste tem como objetivo validar a conexão com o banco de dados da Raspberry, bem como escrita e leitura desses dados. \\ \hline
                \textbf{Insumos para o caso de Teste}    & O servidor Django Raspberry tem que estar ativo. \\ \hline
                \textbf{Roteiro para realização do Teste}&  1. Subir o servidor Django da raspberry, 2.Utilizando o Parser, realizar a conexão com o banco de dados local, 3. Realizar a inserção de um dado em uma tabela, 4. Realizar a leitura deste mesmo dado na tabela, 5. Averiguar se o dado guardado é o mesmo lido  \\ \hline
            \multicolumn{2}{|c|}{\textbf{Cenários de Teste}} \\ \hline
                \textbf{Objetivo Específico}                      & \textbf{Saídas Esperadas} \\ \hline
               Executar inserções no banco			& Dados inseridos \\ \hline
               Tratar caso a conexão com o banco de dados não seja estabelecida & Sistema deverá informar que o banco de dados local está corrompido devido algum problema de hardware/software que venha a ter prejudicado a integridade dos dados \\ \hline
        \end{tabular}
    \end{center}
    \caption[Caso de Teste - CT04]{Caso de Teste - CT04
    \protect Fonte: Autor}
    \label{CT-04}
\end{table}

\begin{table}[H]
    \begin{center}
        \begin{tabular}{|p{5cm}|p{12cm}|}
            \hline
            \multicolumn{2}{|c|}{\textbf{Especificação de Caso de Teste}} \\ \hline
                \textbf{Subsistemas}                               & Interface/Processamento e Eletroeletrônica \\ \hline
                \textbf{Caso de Teste}                             & CT-05 \\ \hline
                \textbf{Propósito}                                     & Testar conexão, escrita e leitura dos dados da simulação do sistema, comunicação entre microcontrolador e Raspberry \\ \hline
                \textbf{Descrição Geral}                           & Este teste tem como objetivo realizar teste de conexão via porta serial entre Raspberry e microcontrolador, utilizando a rotina de comunicação desenvolvida em Python. O sistema deve realizar a leitura e escrita dos dados de forma íntegra. \\ \hline
                \textbf{Insumos para o caso de Teste}    & Subir uma instância de leitura e uma instância de escrita em terminais diferentes, escrever o input da simulação e iniciar um experimento. \\ \hline
                \textbf{Roteiro para realização do Teste}&  1. Subir uma instância de pyserial em modo de leitura, 2. Subir uma instância de pyserial em modo de escrita, 3. Iniciar o experimento com -1 e logo em seguida com a frequência 10, 4. Observa o output do modo de leitura para saber se está de acordo com o que se espera da simulação. \\ \hline
            \multicolumn{2}{|c|}{\textbf{Cenários de Teste}} \\ \hline
                \textbf{Objetivo Específico}                      & \textbf{Saídas Esperadas} \\ \hline
                Executar conexão, escrita e leitura dos dados a partir do microcontrolador & O sistema deverá salvar no banco de dados os dados lidos \\ \hline
                Tratar a perca de dados caso o sistema tenha package loss durante a comunicação & O sistema deverá pedir para o sistema enviar o pacote perdido novamente para leitura e assegurar que o dado chegado é o que se esperava da simuĺação \\ \hline
                Tratar da integridade dos dados obtidos  & O sistema deverá garantir uma integridade contínua dos dados de modo que estes dados tenham confiabilidade. \\ \hline
        \end{tabular}
    \end{center}
    \caption[Caso de Teste - CT05]{Caso de Teste - CT05
    \protect Fonte: Autor}
    \label{CT-05}
\end{table}

\begin{table}[H]
    \begin{center}
        \begin{tabular}{|p{5cm}|p{12cm}|}
            \hline
            \multicolumn{2}{|c|}{\textbf{Especificação de Caso de Teste}} \\ \hline
                \textbf{Subsistemas}                               & Interface/Processamento e Eletroeletrônica \\ \hline
                \textbf{Caso de Teste}                             & CT-06 \\ \hline
                \textbf{Propósito}                                     & Realizar o teste da rotina de requerimento dos sensores ativos no sistema para repasse na aplicação Web \\ \hline
                \textbf{Descrição Geral}                           & Este teste possui uma integridade de alto nível, requerendo um maior cuidado. O objetivo é que a aplicação Web, faça uma requisição dos sensores ativos no sistema, essa requisição será repassada por internet ao BEViM-Rasp até as Rotinas de comunicação que irão iniciar escrita do comando para envio e a leitura dos sensores ativos, que serão lidos pelos métodos do Parser para escrita no banco de dados e por fim buscados por um GET da aplicação Web no banco de dados da Raspberry e atualizados no sistema que o usuário acessa. \\ \hline
                \textbf{Insumos para o caso de Teste}    & Subir os Djangos da Raspberry e da aplicação Web, e preparar a estrutura de comunicação com o Arduino ativo e conectado a Raspberry \\ \hline
                \textbf{Roteiro para realização do Teste}&  1. Subir djangos Web e Raspberry, 2. Ligar Arduino a Raspberry, 3. Repassar a requisição dos sensores ativos via aplicação Web, 4. Verificar se os sensores ativos batem com o que a simulação realmente enviou \\ \hline
            \multicolumn{2}{|c|}{\textbf{Cenários de Teste}} \\ \hline
                \textbf{Objetivo Específico}                      & \textbf{Saídas Esperadas} \\ \hline
                Garantir a funcionaldiade na totalidade das rotinas e do sistema integrado & O sistema deverá responder com os sensores ativos na bancada\\ \hline
        \end{tabular}
    \end{center}
    \caption[Caso de Teste - CT06]{Caso de Teste - CT06
    \protect Fonte: Autor}
    \label{CT-06}
\end{table}

\begin{table}[H]
    \begin{center}
        \begin{tabular}{|p{5cm}|p{12cm}|}
            \hline
            \multicolumn{2}{|c|}{\textbf{Especificação de Caso de Teste}} \\ \hline
                \textbf{Subsistemas}                               &  Interface/Processamento, Eletroeletrônica, Eletroeletrônica e Estrutura \\ \hline
                \textbf{Caso de Teste}                             & CT-07 \\ \hline
                \textbf{Propósito}                                     & Realizar o teste de 10 segundos de um job a uma frequência de 50hz e visualizar seu resultados nos gráficos \\ \hline
                \textbf{Descrição Geral}                           & Este teste têm como objetivo testar a principal função de envio de um comando, aguardo da leitura dos dados no tempo estabelecido e visualização dos dados na interface gráfica. O sistema têm que estar pronto para o grande volume de dados que será enviado a Raspberry, esses dados deverão ser tratados, e guardados na base de dados local, e por fim deverão ser lidos pelo GET da aplicação Web e repassados a interface sobre quais sensores estão ativos. \\ \hline
                \textbf{Insumos para o caso de Teste}    & Subir os Djangos da Raspberry e da aplicação Web, e preparar a estrutura de comunicação com o microcontrolador ativo e conectado a Raspberry e ter a bancada integrada\\ \hline
                \textbf{Roteiro para realização do Teste}&  1. Subir djangos Web e Raspberry, 2. Integrar raspberry com o sistema de controle, 3. Criar 1 job com duração de 10 segundos a um a frequencia de 50hz, 4. Averiguar os dados recebidos e os gráficos que foram montados  \\ \hline
            \multicolumn{2}{|c|}{\textbf{Cenários de Teste}} \\ \hline
                \textbf{Objetivo Específico}                      & \textbf{Saídas Esperadas} \\ \hline
                Tratar a exibição dos dados resultantes do experimento & O sistema web deverá preencher a base de dados com os dados resultantes e a aplicação deverá exibir o gráfico desses dados, bem como realizar o processamento desses dados e exibi-los logo em seguida.  Além disso, a bancada deverá vibrar na frequência estabelecida e o inversor deverá mostrar a frequência. Além disso, será observado se haverá o escorregamento da correia. \\ \hline
        \end{tabular}
    \end{center}
    \caption[Caso de Teste - CT07]{Caso de Teste - CT07
    \protect Fonte: Autor}
    \label{CT-07}
\end{table}

\begin{table}[H]
    \begin{center}
        \begin{tabular}{|p{5cm}|p{12cm}|}
            \hline
            \multicolumn{2}{|c|}{\textbf{Especificação de Caso de Teste}} \\ \hline
                \textbf{Subsistemas}                               &  Interface/Processamento, Eletroeletrônica, Eletroeletrônica e Estrutura \\ \hline
                \textbf{Caso de Teste}                             & CT-08 \\ \hline
                \textbf{Tipo}                                             & Teste de Integração \\ \hline
                \textbf{Propósito}                                     & Realizar teste de 5 jobs com diferentes frequências e diferentes tempos e observar os resultados nos gráficos \\ \hline
                \textbf{Descrição Geral}                           & Este teste têm como objetivo realizar testes de multiplos jobs e frequências para saber se a Raspberry vai aguentar a quantidade de dados, memória e processamento das operações de input  e output em conjunto com a aplicação Web e o sistema de controle simulado pelo Arduino. \\ \hline
                \textbf{Insumos para o caso de Teste}    & Subir os Djangos da Raspberry e da aplicação Web, e preparar a estrutura de comunicação com o microcontrolador ativo e conectado a Raspberry e ter a bancada integrada\\ \hline
                \textbf{Roteiro para realização do Teste}&  1. Subir djangos Web e Raspberry, 2. Integrar raspberry com o sistema de controle, 3. Criar 5 jobs com durações aleatórias e de diferentes frequências, 4. Averiguar os dados recebidos e os gráficos que foram montados  \\ \hline
            \multicolumn{2}{|c|}{\textbf{Cenários de Teste}} \\ \hline
                \textbf{Objetivo Específico}                      & \textbf{Saídas Esperadas} \\ \hline
                Tratar a exibição dos dados resultantes do experimento & O sistema web deverá preencher a base de dados com os dados resultantes e a aplicação deverá exibir o gráfico desses dados, bem como realizar o processamento desses dados e exibi-los logo em seguida.  Além disso, a bancada deverá vibrar nas frequências estabelecidas e o inversor deverá mostrar as frequências. Além disso, será observado se haverá o escorregamento da correia. \\ \hline
        \end{tabular}
    \end{center}
    \caption[Caso de Teste - CT08]{Caso de Teste - CT08
    \protect Fonte: Autor}
    \label{CT-08}
\end{table}

