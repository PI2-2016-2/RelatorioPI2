\chapter{Desenvolvimento da Solução}
\label{desenvolvimento}

Este capítulo apresenta as soluções propostas para o problema definido, de acordo com as frentes de trabalho estabelecidas: Frente Mecânica/Estrutural, Eletromecânica, Eletroeletrônica e Processamento e Interface.

\section{Projeto Mecânico/Estrutural}

\label{desenvolvimento_estrutura}

Nesta seção é descrito o detalhamento da solução e finalização do módulo de mecânica, bem como mudanças que ocorreram e a descrição da integração final com todos os sub-módulos resultando na entrega do produto final previsto na Fase 04.

\subsection{Detalhamento do Módulo}

  O detalhamento da solução para a frente de estrutura pode ser sumarizado nos itens abaixo. A solução completa definida nas Fases 02 e 03 pode ser vista no relatório 2.

    \begin{itemize}
    \item Ocorreram mudanças da carga estática dos pés e da espessura da chapa superior.
    \item Foram feitas análise estruturais e modais da estrutura.
    \item Realizou-se análises modais para o tampo superior com a nova espessura.
    \item Foram efetuados o dimensionamento das molas e assim sucedeu-se a escolha das molas.
    \item Os pés, o motor e as molas foram fixados na estrutura.
    \item Realizou-se um sistema de transmissão entre correia e polias, sendo que uma polia está fixa no motor e a outra situa-se no eixo centralizado em um mancal.
    \end{itemize}

  Outra parte que foi realizada já durante na Fase 03 foi a integração com Eletromecânica conforme é explicada nos tópicos abaixo:

  \begin{itemize}
    \item A integração do subsistema estrutura dá-se diretamente com as frentes eletromecânica e eletroeletrônica. O subsistema interface/processamento não se integra diretamente com a estrutura.
    \item Entre os subsistemas eletromecânica e de estrutura, a integração foi feita através da fixação do motor nos furos dedicados na estrutura e o acoplamento entre a polia do motor ao eixo fixado no mancal, na superfície vibratória da estrutura, que contém a massa desbalanceada, assim a bancada será capaz de vibrar.
  \end{itemize}

  E por fim a parte de eletroeletrônica integra-se a estrutura por meio dos sensores acoplados na superfície vibratória, os quais serão capazes de fazer as medições necessárias em um determinado experimento. Além disso, haverá sensores disponíveis na bancada para serem fixados no corpo de teste.

\subsection{Detalhamento da Integração Final}

Com o objetivo exigir menos potencia do motor e conseguentemente preservar os valores de amplitude e e faixa de frequencia foi proposto o uso de um segundo conjunto de molas com rigidez menor como opção de troca pelo uzuário.

As dimenções físicas do novo conjunto é identico ao anterios porém o material e o diametro do fio que se torna 4mm.

O material é aço inoxidável. E suas caracterpisticas são:

\begin{itemize}
\item m=0,478
\item $A=2911 MPa.mm^m$
\item Aço inoxidável.
\end{itemize}

Os Cálculos são os mesmos apresentados no ponto de controle anterior, logo é possível obter:

$$(K_B)_{inox}=1,2439$$
$$(C)_{inox}=5,875 $$
$$(Fs)_{inox}=102,25kN $$
$$(k)_{inox}=11462 N/m^2$$

Menos rígida que o modelo anterior a mola de aço inoxidável agora exige menos do motor.

%%%%%%%%%% REALIZAR DETALHAMENTO DO TRABALHO DE INTEGRAÇÃO FINAL %%%%%%%%%%%%%%%

% Nesta seção é descrito o detalhamento da solução do módulo de estrutura, resultante das duas primeiras fases,
% e o seu projeto e construção, resultante da Fase 03.

% \subsection{Detalhamento da Solução}

%  O detalhamento da solução para a frente de estrutura pode ser sumarizado nos itens abaixo. A solução completa definida nas
%   Fases 01 e 02 pode ser vista no \href{https://drive.google.com/file/d/0B5InkGKx6O-MR1B3eVYzZFpjQ3c/view?usp=sharing}{Relatório 1}.

%  \begin{itemize}
%   \item A base estrutural da mesa de vibração será composta por cantoneiras de abas iguais. A cantoneira a ser utilizada terá abas iguais e uma bitola de 4,76 x 50,8mm, tendo assim um peso de 3,63kg/m.
%   \item A bancada terá 4 molas fixadas nas extremidades da superfície vibratória e na estrutura da base.
%   \item A bancada terá 4 pés, de borracha e ferro, para que as vibrações fiquem mais distribuídas e não sobrecarreguem apenas uma aresta.
%   \item A mesa de vibração terá uma altura de 900mm e dimensões de 500 x 800mm.
%   \item Foram feitas análises modais levando em consideração o valor do primeiro modo de vibração.
% \end{itemize}
% \subsection{Projeto e Construção}

% \subsubsection*{Mudanças na solução}

% Após iniciado a fase de projeto e construção verificou-se que alguns dos parâmetros decididos nas fases 01 e 02
% deveriam que ser alterados a fim de adequá-los a disponibilidade e a realidade local. As mudanças foram:
% \begin{itemize}
% \item Carga estática dos pés de 50kg  para 70kg
% \item Espessura da chapa de aço de 10 mm para 4.75 mm.

% \end{itemize}

% \subsubsection*{\textbf{Análise Estrutural}}

%     A análise da estrutura da bancada tem como intuito determinar os efeitos das cargas aplicadas sobre a mesma e, assim, os resultados serão usados para examinar se a estrutura está apta para o uso.

%     Essa análise foi feita no software Ansys a partir do design da estrutura projetada no software SOLIDWORKS\footnote{http://www.solidworks.com/}. Os esforços de maiores relevâncias para a estrutura são os pesos da chapa superior e do motor. A partir disso, realizou-se a simulação estrutural colocando como esforços as forças peso da chapa e do motor, conforme indicado na Figura \ref{fig:an_estrutural}.

%   \begin{figure}[H]
%       \centering
%       \includegraphics[scale=0.7]{figuras/an_estrutural.png}
%       \caption{Peso da chapa superior aplicada na estrutura. Fonte: Autores}
%       \label{fig:an_estrutural}
%       \end{figure}

%     A carga relacionada ao peso do motor foi aplicada na estrutura a fim de verificar a deformação máxima que este carregamento pode causar na estrutura.
%     A  Figura \ref{fig:carga_motor} ilustra a carga aplicada através da ferramenta computacional ANSYS\footnote{www.ansys.com/}.


%   \begin{figure}[H]
%       \centering
%       \includegraphics[scale=0.7]{figuras/carga_motor.png}
%       \caption{Peso do motor aplicado na estrutura. Fonte: Autores}
%       \label{fig:carga_motor}
%       \end{figure}

%       Com os esforços definidos, o programa computa automaticamente todos os cálculos estruturais necessários a partir das cargas aplicadas.
%       Os resultados estão mostrados na Figura \ref{fig:def_motor}.

%   \begin{figure}[H]
%       \centering
%       \includegraphics[scale=0.7]{figuras/def_motor.png}
%       \caption{Deformação total devido aos esforços aplicados. Fonte: Autores}
%       \label{fig:def_motor}
%       \end{figure}

%       Os resultados da simulação para deformação máxima da estrutura sob o carregamento do peso do motor são ilustrados pela Figura \ref{fig:result}.

%   \begin{figure}[H]
%       \centering
%       \includegraphics[scale=0.7]{figuras/result.png}
%       \caption{Resultado da deformação - Escala em mm. Fonte: Autores}
%       \label{fig:result}
%       \end{figure}

%       Pelas Figuras \ref{fig:def_motor} e \ref{fig:result}, percebe-se que o peso do motor é o responsável pela maior deformação da estrutura, mas
%       essa deformação é praticamente desprezível por ser extremamente pequena, 0,03 mm. Então a estrutura mostrou-se apta e segura para a utilização.
%     Com o intuito de ter um maior grau de segurança para validar a estrutura, diversas simulações foram feitas aumentando os valores das cargas aplicadas e as
%     deformações continuaram desprezíveis comprovando que a estrutura é válida para o projeto.

%     \vfill
% \subsubsubsection*{\textbf{Análise modal da estrutura}}

%     Uma vez que o projeto trata de uma bancada vibratória, é de suma importância analisar o comportamento da estrutura submetido a vibração. A análise modal da estrutura foi feita no software Ansys e observou-se o primeiro modo de vibração para validar a utilização da mesma.
%     A maior influência vibracional na estrutura é do motor, pelo fato do mesmo estar fixado diretamente na estrutura da mesa. A tampa superior não terá tanta relevância quanto o motor por causa das molas que amorteceram grande parte da propagação da vibração.
%     O primeiro modo de vibração da estrutura está representado na Figura \ref{fig:vib_estrutura} e corresponde a um valor de 65,83 Hz.
%     Analisando a Figura \ref{fig:vib_estrutura}, percebe-se a importância dos reforços horizontais no eixo x, pois a tendência de movimento
%     desse modo de vibração é ao longo do mesmo eixo e os reforços proporcionam uma maior rigidez para esse movimento.

%   \begin{figure}[H]
%       \centering
%       \includegraphics[scale=0.6]{figuras/vib_estrutura.png}
%       \caption{Primeiro modo de vibração da estrutura. Fonte: Autores}
%       \label{fig:vib_estrutura}
%       \end{figure}

%       Dado que o motor tem maior influência na estrutura e mesmo possui uma vibração máxima de 60 Hz e o primeiro modo de vibração foi superior
%       a esse valor, então a estrutura está validada. Para aumentar ainda mais o valor do primeiro modo de vibração, instalou-se pés vibra-stop na estrutura e,
%       assim, o grau de segurança da mesma submetida a vibração foi ampliado.

% \subsubsubsection*{\textbf{Análise modal do tampo}}

%     Tendo em vista que a rotação máxima do eixo para atender as especificações do projeto seja de 6000 rotações por minuto ou 100Hz a
%     plataforma superior da bancada não poderá ter frequência de ressonância menor que 100Hz. Para validação do mesmo foram elaborados ensaios
%     na plataforma ANSYS\footnote{www.ansys.com/} afim de verificar a frequência de ressonância do tampo.

%     A Figura \ref{fig:tampo} mostra que a frequência de ressonância do tampo é superior ao esperado(100Hz) podendo chegar até 106Hz e com isso a
%     validação da escolha deste material. A deformação máxima do tampo corresponde a 13mm.

%   \begin{figure}[H]
%       \centering
%       \includegraphics[scale=0.6]{figuras/tampo.png}
%       \caption{Análise modal do tampo. Fonte: Autores}
%       \label{fig:tampo}
%       \end{figure}

% \subsubsection*{\textbf{Dimensionamento das Molas}}

% Para o presente projeto foram utilizadas quatro molas de comando de válvulas de cabeçote de motor de combustão interna. A escolha desse tipo de mola se deve ao fato de possuir
% vida infinita em uso, sendo, por isso, usada nessa função pela indústria automotiva. Geralmente feitas de liga de Cromo
% e Vanádio, comandos de válvulas giram em torno de 2000 rotações por segundo por longos períodos de tempo,
% qualquer falha durante o uso terá consequências catastróficas para funcionamento do motor. A  Figura \ref{fig:mola}
% ilustra uma mola de comando de válvula utilizada em motor de automóveis.

% \begin{figure}[H]
% \centering
% \includegraphics[scale=0.3]{figuras/mola.png}
% \caption{Mola de comando de válvula. Fonte: Mercedes Rio Diesel}
% \label{fig:mola}
% \end{figure}

% No projeto da mesa vibratória foram usadas 4 molas de comando de cabeçote uma em cada extremidade do tampo. Suas dimensões são:
% \begin{itemize}
% \item Comprimento: 45mm
% \item Diâmetro do fio da mola: 3mm
% \item Diâmetro da mola (diâmtro externo): 23,5mm
% \item Número de espiras ativas 5.
% \end{itemize}

% O uso de molas de comando de válvula, seja em um motor, seja na mesa vibratória do projeto do grupo é sempre em compressão.
% Molas de compressão possuem suas extremidades usinadas reduzindo o número de espiras ativas e contribuindo para armazenar mais energia na mola.
% Um dos principais requisitos do projeto foi a operação da mesa em uma faixa de frequência que se estende entre 70 Hz e 100Hz.
% Sabendo que um Hertz é um ciclo por segundo a máquina opera entre 70 ciclos por segundo e 100 ciclos por segundo daí constata-se que a cada ciclo dura entre:

% $$t_{min} (segundos)=\frac{1segundo}{70 ciclos/segundo}=0.0143$$
% $$t_{max} (segundos)=\frac{1segundo}{100 ciclos/segundo}=0.0100$$

% Para movimento rotativo usa-se, para cálculo de deslocamento da extremidade da mola, a expressão:

% $$deslocamento=amplitude \cos(wt)$$

% Onde t é o tempo e ômega é a velocidade angular da superfície superior da mola por tempo (todos em segundos).

% $$\omega=\frac{2\pi}{t_{ciclo}}$$

% Será usado o tempo para 100 Hz por ser o menor, mais exigente. Logo:

% $$\omega=\frac{2\pi}{t_{max}}=\frac{2\pi}{0,001s}=628,3rad/s$$

% Para encontrarmos a aceleração diferenciamos duas vezes a fórmula do deslocamento:

% $$deslocamento=amplitude * \cos(\omega t)$$
% $$aceleração=-\omega^2 * amplitude * \cos(\omega t)$$

% A aceleração máxima ocorre quando o termo $\cos(\omega t)$ equivale a 1, a situação mais extrema de operação.
% O deslocamento máximo, amplitude do movimento, foi decidido que será de 10 mm (0,01 metro). Com essa informação em mãos pode-se calcular a aceleração máxima:


% $$aceleração =628,3^2 * 0,01 *1 =3948m/s^2 $$

% Tendo em mãos o valor da aceleração é possível calcular a força que cada mola recebe. O tampo pesa exatamente 15.31 kg.
% Como as molas estão a mesma distância do centro de massa do tampo, pode-se afirmar que cada mola apoia um quarto desse valor, 3,83 kg.
% O peso máximo dos elementos que serão estudados sob a mesma são passa de 10kg e devem ser sempre apoiado no centro geométrico da mesma.
% Dessa forma cada mola suporta, devido ao corpo, no máximo, 3.33kg.

% $$ F=m*a=(3,83+3,33)kg*3948m/s^2$$
% $$ F=7,157kg*3948m/s^2$$
% $$F=28257,81kg m/s^2$$
% $$F\approx 28kN$$

% O diâmetro médio (D) de molas é dado pela equação:

% $$D=d_{ext}-d_{fio}$$
% $$D=23,5mm-3mm=20,5mm$$

% Com os valores dos dois diâmetros também é calculado o índice de mola, uma medida adimencional de curvatura da espiral da mola.

% $$C=\frac{D}{d_{fio}}$$
% $$C=\frac{20,5mm}{3mm}=6,834$$

% Econtrando o valor de \textit{Bergsträsser}

% $$K_B=\frac{4C+2}{4C-3}=\frac{4*6,834+2}{4*6,834-3}$$
% $$K_B=1,2054,$$

% Pode-se agora calcular a força limite, por segurança, para a mola atingir seu comprimento sólido. Ou seja, a força que,
% se exercida sob a mola, ocasionará contato entre suas espiras.

% $$F_s=\frac{\pi d^3 \alpha}{8 K_B D}$$ onde $$\alpha=\frac{S_{sy}}{n_s}$$

% Fs depende do material que é utilizado para confeccionar a mola, no caso de molas de comando de válvula, como informado,
% são feitas de Ligas de Cromo e Vanádio.

% Em teoria de projeto de molas a tensão última de um material pode ser encontrada por uma relação entre o seu diâmetro,
% um expoente tabelado e uma característica físicas, pela fórmula:

% $$S_{ut}=\frac{A}{d^m}$$

% Para o Cromo Vanádio tem-se, de acordo com Shigley (Projeto de Engenharia Mecânica 7ª Edição, pg 496 e 497 as características apresentadas na Tabela \ref{tab:caracteristicas_cromovanadio}.

% \begin{table}[H]
%     \begin{tabular}{|p{2cm}|p{2cm}|p{2cm}|p{2cm}|p{2cm}|p{2cm}|}
%         \hline
%         \textbf{Material} & \textbf{Número ASTM} & \textbf{Expoente} & \textbf{Diâmetro do fio (mm)} & \textbf{A (MPa $mm^{mm}$)} & \textbf{G(GPa)} \\ \hline
%         Fio de cromo-vanádio& A232& 0,168& 3&2005 &77,2                                                 \\ \hline
%     \end{tabular}
%     \caption{Características físicas do aço Cromo-Vanádio. Fonte: \cite{shigley}}
%     \label{tab:caracteristicas_cromovanadio}
% \end{table}


% Contendo assim:

% $$S_{ut}=\frac{2005}{3^{0,168}}=1667,1MPa$$

% \textit{Shigley} também afirma que, para molas de cromo-vanádio:

% $$S_{sy}=S_{ut}*0,5$$
% $$S_{sy}=1667,1*0,5=S_{sy}=833,5MPa$$

% Utilizando fator de segurança de 1,2:

% $$\alpha=\frac{833,5MPa}{1,2}=694,6MPa$$

% Retomando a fórumula:

% $$F_s=\frac{\pi d^3 \alpha}{8 K_B D} = \frac{\pi * 3mm^3 * 694,6MPa}{8*1,2054 * 20,5mm}=298,01kN$$

% Esse valor define a dimensão máxima da força que cada mola pode receber para que suas expiras se toquem. Sabendo que cada mola na mesa recebe 28kN de força, conclui-se:
% \begin{itemize}
% \item As molas suportam o limite máximo de carga sub a mesa: 10kg
% \item Sob frequencia máxima definida de 100Hz.
% \item Com fator de segurança muito alto: $n=\frac{F_S}{28kN}$
% \end{itemize}

% A rigidez (k) ou razão de mola é um parâmetro muito importante no projeto de uma mola mecânica. A teria informa que:

% $$k \frac{d^4G}{8 D^3N_a}$$

% A constante Na refere-se ao número de espiras ativas na mola, 5 e G é um valor tabelado conhecido do material da mola. No caso de fio de cromo-vanádio, G=77,2GPa

% $$k=\frac{3mm^4 77,2GPa}{8*20,5mm^3*5}=18145N/m^2$$

% Essa informação é importante para as simulações dinâmicas da mesa e tampo em softwares computacionais.  Mas é principalmente usada em considerações sobre frequência de ressonância dentro das situações de uso. Para o projeto da mesa, frequências de uso entre 70Hz e 100Hz. A fórmula a seguir apresenta, em função da razão de mola, as frequências que uma mola entra em ressonância se excitada entre placas paralelas.

% $$\omega=m\pi \sqrt{\frac{k*g}{W}}$$

% g é a aceleração da gravidade, W o peso da mola e m é o número do harmônico calculado, sendo m=1 o a frequência fundamental.

% $$W=\frac{\pi^2 d^2 D N_a \gamma}{4}$$
% $$\gamma=7,86*10^{-3}kg/mm^3$$
% $$W=17,8897N$$
% $$\omega=1\pi \sqrt{\frac{0,018145N/mm 9180mm/s^2}{17,8897}}$$

% A mola empregada na construção da mesa passa bem pelas exigências do projeto, respeitando a os requisitos de carga e frequência de utilização, já que o seu primeiro harmônico é superior ao 109 va lor máximo de 100Hz requerido pelo projeto.

% \subsubsection*{\textbf{Fixações}}

%     Tendo em vista que a problemática do projeto consiste em produzir vibrações, é de suma importância uma boa projeção das fixações que envolvem o projeto como um todo. Tais fixações devem ser atentadas para que não haja falhas ou até mesmo perda de funcionalidade de componentes. Nos tópicos seguintes são discutidas como serão as fixações bem como suas respectivas justificativas.

% \subsubsubsection*{\textbf{Fixação dos pés}}

%     Os pés de borracha que sustentam a estrutura do projeto consistem em fixação por parafuso e porca. Para que esta fixação não comprometa a integridade da estrutura, foram elaborados apoios com furação iguais aos das roscas dos pés, as quais são de meia polegada. Além disso, como o projeto consiste em constantes vibrações sobre a estrutura, foram realizados travamento para que as porcas não venham folgar ou até mesmo soltar ao longo do tempo.
%     Tendo em vista este problema a fixação é feita por duas porcas na parte superior da estrutura, como podemos ver na Figura \ref{fig:config_pes}. A fim de ajustar erros de construção obtidos pela solda da estrutura a qual provocou desnível na mesma foram colocadas arruelas na medida certa até que houvesse o nivelamento da estrutura.

%     \begin{figure}[H]
%       \centering
%       \includegraphics[scale=0.4]{figuras/config_pes_jpg.png}
%       \caption{Configuração da fixação dos pés. Fonte: Autores}
%       \label{fig:config_pes}
%       \end{figure}

% \subsubsubsection*{\textbf{Fixação do motor}}

%     O motor de indução trifásico do tipo gaiola o qual poderá exercer frequências de até 60Hz sobre a estrutura é fixado a partir de parafusos, arruelas de pressão e porcas. Como a constância de vibração do motor exercido sobre a estrutura é alta a necessidade de colocar coxins entre a carcaça do motor e a estrutura foi um ponto crucial para o projeto, absorvendo assim as vibrações do motor para com a estrutura. Para que as porcas não venham a folgar ou até mesmo soltar durante o funcionamento do motor foram colocadas arruelas de pressão, garantindo assim com que o conjunto parafuso e porca fique fixo e não se soltem a longo prazo.

% \subsubsubsection*{\textbf{Fixação das molas}}

%     As molas são os componentes mais delicados do projeto, tendo em vista que será o componente que mais sofrerá esforço. Com base nisto é de suma importância estar atento a todos os detalhes deste item do projeto. Para que não haja falha das molas, é de entendimento que tais componentes não podem ser soldados na estrutura, para que o projeto atenda tais requisitos as molas são fixadas na estrutura através de interferência do diâmetro interno da mola para com copinhos de fixação que foram projetados pelos próprios autores. Para que não houvesse solda entre os copinhos com a estrutura foram elaboradas chapas de sustentação dos copinhos, nos quais tem furação para que a solda fosse elaborada por baixo dos copinhos.
%     Para melhor entendimento a Figura \ref{fig:config_copinhos} ilustra a configuração dos copinhos.

% \begin{figure}[H]
% \centering
% \includegraphics[scale=0.5]{figuras/config_copinhos.png}
% \caption{Configuração das molas. Fonte: Autores}
% \label{fig:config_copinhos}
% \end{figure}

% \subsubsection*{\textbf{Sistema de Transmissão}}

% \subsubsubsection*{\textbf{Polias e correias}}

%     As polias são peças cilíndricas, movimentadas pela rotação do eixo do motor e pelas correias. Os tipos de polia são determinados pela forma da superfície na qual a correia se assenta. Correias são elementos de máquinas que transmitem movimento de rotação entre dois eixos (motor e movido) por intermédio de polias. Elas são empregadas quando se pretende transmitir potência de um veio para o outro a uma distância em que o uso de engrenagens é inviável. Para o sistema em questão será usada as polias do tipo Trapezoidal ou V múltipla.

%     A correia em V ou trapezoidal é inteiriça, fabricada com seção transversal em forma de trapézio. É feita de borracha revestida de lona e é formada no seu interior por cordonéis vulcanizados para suportar as forças de tração. O emprego da correia trapezoidal ou em V é preferível ao da correia plana pelos seguintes motivos: Praticamente não apresenta deslizamento; Permite o uso de polias bem próximas; Elimina os ruídos e os choques, típicos das correias emendadas (planas).

%     Como as correias têm características diferentes de fabricante para fabricante, é aconselhável seguir as instruções que eles forneçam. A partir destes elementos pretende-se selecionar a polia do tipo guia com correia do tipo trapezoidal a ser usada observando o tipo, a secção e o comprimento primitivo, potência a ser transmitida, tipos de máquina motoras e movidas, velocidade angular da polia motora e da polia movida, distância entre os eixos da polia motora e da polia movida, distância entre os eixos das polias, na qual o comprimento máximo admitido deve ser igual a três o produto da soma dos diâmetros da polia motora e movida e finalmente o tipo de carga(uniforme, choques moderados, choques intensos).

%     A polia maior acoplada na saída do eixo do motor tem diâmetro de 250mm, feita de alumínio fundido e é fixada na ponta de eixo do motor através de chavetas e parafusos. A Figura\ref{fig:config_polia} ilustra o esboço do acoplamento do motor com a polia e da polia com a correia.

% \vfill

% \begin{figure}[H]
% \centering
% \includegraphics[scale=0.6]{figuras/config_polia.png}
% \caption{Configuração do acoplamento polia, motor e correia. Fonte: Autores}
% \label{fig:config_polia}
% \end{figure}

% \subsubsubsection*{\textbf{Mancal e eixo}}

%     Fora escolhido como mecanismo de transmissão de movimento deste projeto, mancal e eixo centralizado. Mancal é um componente de uma máquina que possui a função de permitir com que o eixo flutue em uma posição determinada, ou seja, funciona como um suporte ou guia a fim de que o eixo possa ser utilizado sem perdas de desempenho devido ao contato com peças externas.

%     A escolha do mancal deve-se ao fato de suas diversas vantagens que se enquadram nos pré-requisitos do projeto tais como: amortecem as vibrações, choques e ruídos, construção simples e custos menores nos projetos dependendo da utilização e finalidade. Os mancais podem ser separados em dois grupos: os de rolamento e os de deslizamento. O que fora escolhido no projeto foi o mancal de deslizamento devido ao fato dele adaptarem-se facilmente as circunstâncias, simples de montar e desmontar e de já o possuirmos.

%     Ao se escolher os mancais de deslizamento houve a preocupação de tomar alguns cuidados em relação a sua manutenção, esses são sujeitos as forças de atrito estas por sua vez surgem devido a rotação do eixo que exercerá carga nos apoios, ou seja, deve-se haver um sistema de lubrificação com o intuito de minimizar as perdas pelo atrito

%     Os mancais de deslizamentos são constituídos de bucha (corpo cilíndrico oco que envolve o eixo) fixada num suporte, estes mancais são utilizados em máquinas pesadas ou em equipamentos com pouca rotação, pois devido ao atrito os componentes aquecem. Ao se utilizar a bucha e lubrificantes reduzimos o atrito e melhoramos a rotação e consequentemente melhoramos o desempenho da rotação.

%     O objetivo em se utilizar este sistema é colocar uma massa desbalanceada com o intuito de que gere na mesa as vibrações desejadas, para que isso ocorre tem que levar em consideração tais observações: Massa desbalanceadora, que se trata de um peso com uma distribuição não uniforme por meio deste podemos obter amplitudes de vibração ou seja quanto maior for a massa desbalanceada maior será a amplitude de vibração; Raio da ação da massa desbalanceadora, pois quanto maior for o raio desta massa maior será a amplitude para a mesma massa desbalanceada; Rotação da polia, em relação a esta podemos analisar que ao se aumentar a rotação aumenta-se a amplitude de vibração de acordo com o desbalanceamento. A Figura \ref{fig:mancal} ilustra melhor o mancal e seu respectivo eixo a ser utilizado no projeto.

% \begin{figure}[H]
% \centering
% \includegraphics[scale=0.9]{figuras/mancal.jpg}
% \caption{Mancal a ser utilizado com seu respectivo eixo e polia acoplados. Fonte: Autores}
% \label{fig:mancal}
% \end{figure}

% \subsubsubsection{\textbf{Relação polia/eixo}}

%     Tendo em vista que a rotação máxima do motor é de 1720 rotações por minuto e que esta rotação equivale apenas a 28,67Hz. Como uma das primícias do projeto visa chegar à vibrações de até 100Hz é necessário aumentar a rotação do motor. Para tal utilizamos uma relação entre polias para que seja possível chegar uma rotação de no mínimo 6000 rotações por minuto o que equivale a uma relação de 3,5. Tendo em vista a polia maior com diâmetro de 250mm é necessário que a polia menor seja de diâmetro máximo de 71,42mm. Entre as polias comerciais a menor mais próxima do valor citado anteriormente equivale a um diâmetro de 60mm atendendo assim as especificações do projeto, podendo assim o eixo fixado na mesa chegar até 7166 rotações por minuto.


\section{Projeto Eletromecânico}

 \label{desenvolvimento_eletromecanica}
Nesta seção é descrito o detalhamento da solução do módulo de eletromecânica, resultante das duas primeiras fases,
e o seu projeto e construção, resultante da Fase 03.

\subsection{Detalhamento da Solução}

O detalhamento da solução para a frente de eletromecânica pode ser sumarizado nos itens abaixo. A solução completa definida nas
  Fases 01 e 02 pode ser vista no \href{https://drive.google.com/file/d/0B5InkGKx6O-MR1B3eVYzZFpjQ3c/view?usp=sharing}{Relatório 1}.

\subsubsection*{Considerações iniciais}

Para a construção de uma bancada com capacidade de vibração mecânica e análise de parâmetros, é de interesse um sistema de estudos para comportamento mecânico frente às vibrações, sendo um equipamento para instalações fixas com limites de carga e dimensões para sistemas analisados. Portanto, para análise eletromecâmica, seguem os estudos realizados abaixo.

\begin{figure}[h!]
	\centering
		\includegraphics[keepaspectratio=true,scale=0.6]{figuras/1.png}
	\caption{Bancada para ensaio de teste de vibração mecânica. Fonte: Autoria Própria}
    \label{bancada}
\end{figure}

As funções principais do controle de um motor são: partida, parada, direção de rotação, regulação da velocidade, limitação da corrente de partida, proteção mecânica e proteção elétrica. Um motor só começa a girar quando o momento de carga a ser vencido, quando parado, for menor do que seu conjugado de partida. Em determinadas aplicações há necessidade de uma rápida desaceleração do motor e da carga.

\begin{figure}[h!]
	\centering
		\includegraphics[keepaspectratio=true,scale=0.6]{figuras/2.png}
	\caption{- Motor de indução trifásica. Fonte: WEG}
    \label{motor}
\end{figure}

O modo utilizado para variação de velocidade do motor de indução será através do inversor de frequência, o qual possibilita o controle do motor CA variando a frequência, mas também realiza a variação da tensão de saída para que seja respeitada a característica V/F ( Tensão / Freqüência) do motor.

No projeto em questão, será usado um motor de indução trifásico conectado a um eixo preso por mancais soldados no tampo da mesa, usando-se para isto uma correia encaixada na polia fixa no eixo do motor e no eixo preso nos rolamentos dos mancais. Com isso, tem-se o intuito de acionar o sistema, gerando assim uma vibração. Para realizar esta vibração, a velocidade do motor será alterada com o uso de um inversor de frequência e com a razão de voltas da polia fixa no motor e o do eixo fixo nos mancais. O conjunto motor e inversor dimensionado deverá atender o torque solicitado e as velocidades solicitadas por todo o sistema.

	Para avaliação térmica do motor realizado por meio da tecnologia do termovisor, na qual fornece uma avaliação da temperatura para com a classe de isolamento, na qual podemos observar na figura \ref{camera} abaixo o comportamento térmico do mesmo para a classe de isolamento do tipo F.

\begin{figure}[h!]
    \centering
	\includegraphics[keepaspectratio=true,scale= 0.5]{figuras/camera.jpg}
	\caption{- Câmara térmica. Fonte: Autoria própria }
    \label{camera}
\end{figure}

No projeto em questão optamos pela ligação do motor em estrela, dada que a disponibilidade da rede elétrica local é 380 V.

\begin{figure}[!h]
	\centering
		\includegraphics[keepaspectratio=true,scale=1.0]{figuras/3.png}
	\caption{Inversor de frequência elegido. Fonte: Allen Bradley}
    \label{inversor}
\end{figure}

O inversor de frequência trabalha com rampas para partida e frenagem. Na partida é utilizada a rampa de aceleração, a velocidade inicia em zero e atinge a velocidade desejada, podendo ajudar o tempo numa faixa de milésimos de segundo. A frequência do rotor é maior do que a frequência do estator durante a frenagem, dessa forma é provocado um fluxo reverso da energia do rotor diretamente ao estator (COVINO; GRASSI; PAGANO, 1997). A rampa de desaceleração é responsável por controlar a frenagem, esse processo se dá pela redução de forma controlada da frequência aplicada ao motor. A figura \ref{rampas} abaixo evidencia as rampas de aceleração e desaceleração.

\begin{figure}[h!]
	\centering
		\includegraphics[keepaspectratio=true,scale=0.9]{figuras/4.png}
	\caption{Rampas de aceleração e desaceleração geradas pelo inversor de frequência. (SIEMENS, 2006)}
    \label{rampas}
\end{figure}

\subsubsection*{Medidas construtivas}

Na primeira parte do trabalho, basicamente realizou-se um estudo sobre o tema proposto, definiu-se os conceitos básicos, realizou-se os cálculos iniciais para definição das diretrizes a serem tomadas e definiu-se qual o motor e o inversor a ser usado. Após a realização do ponto de controle 1, foram tomadas as seguintes medidas construtivas:

\begin{itemize}
\item Aquisição do motor de indução trifásico de 0,5 cv;
\item Aquisição do inversor de frequência da marca Bradley;
\item Aquisição de cabos trifásicos de 2,5cm de diâmetros, valor calculado de acordo com a norma 5410;
\item Aquisição de tomadas trifásicas com capacidade de 10 A;
\item Realização do cabeamento do motor, sendo este ligado em estrela de acordo com o indicado em sua placa de identificação;
\item Realização do cabeamento do inversor de frequência, sendo um cabeamento para ligação na fonte de alimentação e o outro para conexão com o motor de indução;
\item Realização de testes no motor, verificando que o mesmo encontra-se em perfeitas condições de funcionamento;
\item Realização de medição da corrente em cada fase do motor;
\item Realização de teste de temperatura no motor de indução, com o auxílio de uma câmera térmica;
\item Estudo do inversor de frequência para conhecimento de seus parâmetros de funcionamento;
\item Parametrização do inversor afim de satisfazer a necessidade do projeto;
\item Realização de testes de partida e frenagem do motor, usando-se o inversor de frequência;
\item Fixação do motor de indução na estrutura da bancada;
\item Definição do sistema de proteção.
\end{itemize}


A tabela \ref{tabela_info_motor} mostra os dados elétricos do motor de indução escolhido.

        \begin{table}[h]
            \begin{center}
              \begin{tabular}{|p{5cm}|p{5cm}|}
                \hline
                \textbf{Dados elétrico do motor} &
                \\ \hline
                Carcaça & 71
                \\ \hline
                Potência & 0,5HP
                \\ \hline
                Frequencia & 60Hz
                \\ \hline
                Polos & 4
                \\ \hline
                Rotação Nominal & 1720 RPM
                \\ \hline
                Escorregamento & 4,44
                \\ \hline
                Tensão Nominal & 220/380 V
                \\ \hline
                Corrente Nominal & 2.07/1.20 A
                \\ \hline
                Corrente de Partida & 10.45/5.99 A
                \\ \hline
                LP/LN & 5.0
                \\ \hline
                Corrente a vazio & 1.50/0.868 A
                \\ \hline
                Conjugado Nominal & 2.06 Nm
                \\ \hline
                Conjugado de partida & 240
                \\ \hline
                Conjugado Máximo & 250
                \\ \hline
                Categoria & N
                \\ \hline
                Classe de isolação & B
                \\ \hline
                Elevação de temperatura & 80 K
                \\ \hline
                Tempo de rotor bloqueado & 18 s (quente)
                \\ \hline
                Fator de serviço & 1.15
                \\ \hline
                Regime de serviço & S1
                \\ \hline
                Temperatura ambiente & -20 graus Celsius - +40 graus Celsius
                \\ \hline
                Altitude & 1000 m
                \\ \hline
                Proteção & IP55
                \\ \hline
                Massa aproximada & 10 Kg
                \\ \hline
                Momento de inércia & 0,00082 Kg
                \\ \hline
                Nível de ruído & 47 dB(A)
                \\ \hline
              \end{tabular}
              \caption[Informações detalhadas sobre o motor]{Informações detalhadas sobre o motor
              \protect Fonte: CATÁLOGO WEG }
            \label{tabela_info_motor}
        \end{center}
    \end{table}

\subsection{Projeto e Construção}

\subsubsection*{Parametrização do inversor de frequência}

Após a realização do estudo acerca do inversor de frequência, foi realizada a parametrização do mesmo. Com base no manual disponibilizado pela NHP \textit{Eletrical Engineering Products}, uma empresa australiana renomada no setor da engenharia elétrica, o controle do motor foi minimamente estabelecido em todos os detalhes necessários para atender a demanda do projeto.

O inversor dispõe de uma interface bastante amigável, como pode ser visto na figura \ref{Interface do inversor}

\begin{figure}[h!]
	\centering
		\includegraphics[keepaspectratio=true,scale=0.6]{figuras/interface_inversor.png}
	\caption{Interface do inversor de frequência. Fonte: NHP}
    \label{Interface do inversor}
\end{figure}

Para a programação do inversor, foi utilizado apenas os parâmetros básicos, visto que, dessa forma, a nossa necessidade seria atendida. A figura \ref{Parâmetros do inversor} mostra os parâmetros utilizados para fazer o controle do motor de indução através do inversor de frequência.

\begin{figure}[h!]
	\centering
		\includegraphics[keepaspectratio=true,scale=0.9]{figuras/parametros_inversor.png}
	\caption{Listagem dos parâmetros básicos utilizados. Fonte: NHP}
    \label{Parâmetros do inversor}
\end{figure}

No parâmetro P031 foi ajustada a tensão na qual o motor necessita, no caso a tensão foi ajustada em 380V. No parâmetro P032 foi ajustada a frequência nominal do motor de indução, fixada em 60Hz. O parâmetro P033 não foi alterado. Os parâmetros P034 e P035 remetem, respectivamente, às frequências mínima e máxima, ajustadas em 0 e 60Hz. O parâmetro P036 estabelece de que forma o start ocorrerá, no caso foi selecionada a opção 0 que estabelece que será pelo botão na interface do inversor de frequência. O parâmetro P037 não foi alterado. Já o parâmetro P038 foi ajustado na opção 2, cuja estabelece o controle via entrada de 0 a 10V, nessa parte acontece a integração com a frente de eletroeletrônica, onde haverá o controle estabelecido por eles. Os parâmetros P039 e P040 estabelecem os tempos de aceleração e desaceleração, respectivamente, esses tempos estão definidos em valores por volta de 2 a 4 segundos. Os últimos três parâmetros não foram alterados.

\subsubsection*{Sistema de Segurança}

Sistemas de segurança elétrica são concebidos com a finalidade de garantir uma operação segura para os usuários finais, equipes de manutenção e dos próprios equipamentos, que em casos de ocorrências internas ou externas os danos são minimizados.

O projeto inicial concebia a utilização além dos disjuntores e fusíveis, a utilização de dispositivos anti-surto, no sistema trifásico bem como um diferencial residual, no entanto para fins de minimização de custos, considerando a fase de protótipo, foram considerados apenas a utilização de fusíveis na entrada principal trifásica e disjuntores com função de desarme e dispositivo do manobra nos diferentes circuitos que são: Inversor-motor, controle eletrônico, e tomadas auxiliares, dispostos segundo o diagrama mostrado na figura \ref{Diagrama de Instalações}.

\begin{figure}[h!]
	\centering
		\includegraphics[keepaspectratio=true,scale=0.6]{figuras/Diagrama_Instalacao.png}
	\caption{Diagrama de Instalações e Sistema de Segurança}
    \label{Diagrama de Instalações}
\end{figure}

\section{Projeto Eletroeletrônico}

\label{desenvolvimento_eletroeletronica}

  \subsection{Sensoriamento e disponibilização dos dados}
  \subsubsection{Requisito}
  Uma mesa de vibração necessita detalhes específicos sobre as forças que estão sendo aplicadas no decorrer do experimento. Na prática, isso implica em uma grande oferta de sensores capazes de cumprir com taxas de amostragem altas. A frequência na qual os sensores devem amostrar os dados pode ser estimada utilizando o teorema de \textit{Nyquist-Shannon}, que estabelece que frequências de amostragem em sinais contínuos devem ser pelo menos maiores que o dobro da sua frequência. Trazendo isso para o escopo do projeto, que funciona de 50 a 100Hz, teremos no pior caso que amostrar a base à no mínimo 200Hz. Com leituras contínuas nessa velocidade, evitamos o efeito \textit{Aliasing}, que é a perda do formato da onda devido a baixa amostragem. Também cabe dizer que quanto mais próximo de um movimento senoidal de frequência única, melhor. Isso pois o processo acaba sendo melhor modelado em alguns parâmetros bem definidos: Amplitude, Frequência, Velocidade e Aceleração. Mas, em todo caso, frequências de ordens superiores ainda podem estar presentes, inviabilizando a obtenção da leitura. Nesse caso, a estrutura deve ser preenchida por filtros passa-baixas passivos que atuem de forma a eliminar estes problemas.Por fim, o alvo dos testes, seja uma estrutura, seja um outro dispositivo, também necessita ser validado em alguns aspectos. Primeiro, o seu peso não pode ultrapassar o limite seguro para a estrutura começar a rotina de testes, do contrário há muita propensão a acidentes ou a danos estruturais à bancada. Da mesma forma, deve-se ter o \textit{feedback} de alguns pontos chaves do objeto medido, para se ter ideia da vibração no corpo a ser medido.
  \subsubsection{Implementação}
Optou-se por um esquema de barramento para obter dados de sensores, pois há a facilidade de conectar uma grande quantidade deles e compartilhar a mesma rede. Neste caso, optou-se pela rede I2C (\textit{Inter Intergrated Circuits}), onde todos os periféricos recebem 2 sinais de comunicação (Sinal de dados ou \textit{SDA} e Sinal de \textit{clock}, ou \textit{SCL}). A solução permite ter diversos sensores individuais sem a necessidade de um aumento de infraestrutura. Para garantir que a taxa de amostragem do sistema esteja dentro dos requisitos expostos anteriormente, será utilizado um microcontrolador dedicado para essa tarefa. Para a identificação da vibração na mesa foram escolhidos sensores do tipo Acelerômetros. Estes são especialmente desenvolvidos para aferir aceleração em eixos distintos.
\subsubsection{Como acelerômetros funcionam}
Grande parte dos acelerômetros trabalham a partir dos princípios dos efeitos piezoelétricos, que ocorrem quando cristais são excitados e consequentemente geram uma diferencia de potencial elétrico.  Acelerômetros piezoelétricos trabalham usando a segunda lei de newton (\textit{Força = Massa X Aceleração}). A aceleração do objeto em questão a ser avaliado é transmitida para uma massa sísmica dentro do acelerômetro, essa massa gera uma forca proporcional em um cristal piezoelétrico que consequentemente gera uma tensão elétrica proporcional a aceleração. Resumindo, a massa converte a aceleração a ser medida em uma forca, e o cristal converte essa forca em uma tensão elétrica que indica a aceleração em questão
\subsubsection{Parâmetros importantes de serem avaliados na escolha de um acelerômetro}
\begin{itemize}
\item Amplitude de vibração: Se a amplitude de vibração for superior a especificação do acelerômetro o mesmo ira grampear o sinal de saída ou ate provocar uma distorção. Geralmente o aumento da amplitude resulta em uma menor sensibilidade para o sensor. 
\item Sensibilidade: Talvez o parâmetro mais importante, descreve a conversão entre vibração e tensão elétrica, geralmente é dada por \textit{mV/G} (onde \textit{G} é a aceleração da terra, 9.81 $m/s^2$ ). A sensibilidade de um sensor de aceleração também é afetada pela frequência de vibração, o dispositivo é apto a operar dentro de um intervalo de frequência determinado pelo fabricante.
\item Frequência de Vibração: Como citado anteriormente, os sensores de aceleração possuem intervalos de frequência que estão aptos a operar. Se essa frequência for muito baixa é possível que o equipamento nem a detecte e se for muito alta é provável que a tensão de saída sature ou até mesmo que o sensor seja danificado.
\item Numero de eixos: Os tipos mais simples de acelerômetros são uniaxiais, isto é, medem a aceleração para um único eixo orientado dependendo da maneira com que o acelerômetro esta montado na estrutura. Existem também acelerômetros biaxiais e triaxiais, os acelerômetros triaxiais geram três sinais tensão que geralmente são usados para determinar o tipo de vibração seja ela lateral, transversal ou rotacional.
\end{itemize}
\subsubsection{Medições com acelerômetros}
\begin{itemize}
  \item Aceleração e amplitude de vibração: Naturalmente dadas pela tensão elétrica gerada na saída do sensor, porem costuma-se ser mais usado o valor \textit{RMS} (\textit{Root Mean Square}) do sinal para que uma noção melhor da energia de vibração possa ser interpretada.
  \item Velocidade: Sabendo que a velocidade pode ser calculada pela integral da aceleração, para extrair a velocidade de vibração de um sistema/estrutura em questão a solução consiste em integrar o sinal de aceleração e assim obter a velocidade.
  \item Deslocamento: Conhecer o quanto que uma estrutura se desloca enquanto vibra é um parâmetro muito importante em uma analise vibratória. Sabe-se que o deslocamento em função do tempo é dado pela dupla derivada da aceleração no tempo, assim usando o parâmetro de aceleração que o acelerômetro nos da se torna possível analisar o deslocamento da estrutura.
  \item Frequência: É um parâmetro que pode ser difícil de ser analisado pois depende que a vibração com qual a estrutura é excitada seja próxima de constante durante um breve período de tempo, quanto maior esse intervalo de tempo maior sera a precisão dessa medição. A frequência costuma ser obtida atraves de algorítimos de processamento do sinal de aceleração.
\end{itemize}
\subsection{Controle de Malha Fechada}
\subsubsection{Requisito}
É importante que o sistema seja capaz de garantir que os parâmetros de vibração inseridos pelo usuário em camadas superiores do sistema sejam efetivamente o que acontece fisicamente, para implementar essa funcionalidade será usado um sistema de controle por malha fechada. Esse procedimento pode ser representado pela imagem abaixo: 
%imagem 1
O que sistema de controle eletroeletrônico faz é receber um valor de frequência das camadas superiores de interface/processamento e converter essa informação em sinais elétricos. Esses sinais elétricos serviram para controlar o mecanismo vibrador que irá excitar a plataforma vibratória. Na plataforma vibratória existiram sensores que iram informar ao sistema de controle qual a vibração que os sinais elétricos estão provocando. O sistema de controle deverá através desse sinal de retorno avaliar se os sinais elétricos que o mesmo está produzindo estão coerentes com o que o sistema de Interface/Processamento lhe informou anteriormente. O mecanismo vibrador será constituído por um inversor de frequência que controla a velocidade de um motor trifásico que irá através de um sistema mecânico provocar a vibração na mesa. A programação do inversor de frequência e o controle do motor não é realizado pelo sistema eletroeletrônico, o sistema apenas fornece níveis de tensão analógica e sinais digitais para informar o inversor a velocidade com qual o motor deve vibrar.
\subsubsection{Implementação}
A figura a seguir ilustra como será o sistema eletrônico planejado para o controle em frequencia. No projeto, existem 2 microcontroladores atuando em conjunto. O primeiro, e principal, é o MCU 1. Ele garante algumas coisas no sistema:
\begin{enumerate}
\item Velocidade de amostragem correta
\item Comunicação com \textit{Raspberry Pi} via \textit{UART}
\item Comunicação com a interface do motor via \textit{$I_{2} C$}
\end{enumerate}
A segunda parte do projeto é a MCU2, responsável pela manutenção do requisito de vibração na mesa. A partir das informações contidas no barramento, a mesma irá aumentar ou diminuir a velocidade dos motores (a partir de uma interface analógica com o motor), atuando como um controlador PI. As informações como frequência almejada, velocidade atual e frequência atual serão enviadas compartilhadas pelo barramento.
%imagem2

\section{Projeto de Processamento e Interface}

\label{desenvolvimento_processamento}

Nesta seção é descrito o detalhamento da solução do módulo de interface/processamento, resultante das duas primeiras fases,
e o seu projeto e construção, resultante da Fase 03.
Além disso, a visão da solução e da arquitetura estão documentadas nos Apêndices \ref{documento_visao} e \ref{documento_arquitetura}, respectivamente.

\subsection{Detalhamento da Solução} \label{software:detalhamento_solucao}

A solução completa definida nas Fases 01 e 02 pode ser vista no \href{https://drive.google.com/file/d/0B5InkGKx6O-MR1B3eVYzZFpjQ3c/view?usp=sharing}{Relatório 1}.

A Figura \ref{fig:arquitetura_solucao} apresenta a solução proposta como um todo e nela podemos destacar a atuação da
frente de interface/processamento nos seguintes subsistemas de software:

\begin{itemize}
 \item \textbf{Interface de comunicação entre o usuário e a bancada, realizada por meio da aplicação web;}
      \subitem Na aplicação web o usuário poderá controlar a bancada, iniciando um ensaio com os quesitos definido por ele,
	       e visualizar os resultados do ensaio por meio de gráficos.
 \item \textbf{Servidor de Controle para intermediar a comunicação da aplicação web com o sistema de controle eletrônico, hospedado na \textit{Raspberry}.}
      \subitem Este servidor provê serviços RESTFul (API REST) como a interface de comunicação para a aplicação web. Neste servidor estão as 
	       rotinas de comunicação com o sistema de controle eletrônico, que realizam a inserção de dados de comando e recupera os dados 
	       dos sensores via comunicação serial.
\end{itemize}

A interação entre esses dois subsistemas que compõem o módulo de interface/processamento funcionará da seguinte forma
(o protocolo de comunicação definido será apresentado na subseção \ref{software:protocolo}):

\begin{enumerate}
 \item O usuário cadastrado autentica-se no sistema (caso o usuário não seja cadastrado basta ele se cadastrar);
 \item O usuário monta um ensaio com seus respectivos \textit{jobs} (o conceito de \textit{job} é apresentado na subseção \ref{software:app_web});
 \item O usuário confere os sensores ativos identificados pelo sistema e inicia o experimento;
    \subitem O usuário tem a possibilidade de atualizar os sensores ativos, caso perceba erro na leitura dos sensores;
    \subitem Quando o usuário solicita a identificação dos sensores ativos, a aplicação web envia uma requisição
         de identificação dos sensores ativos ao Servidor de Controle, seguindo o protocolo definido.
 \item A aplicação web envia uma requisição ao Servidor de Controle informando os dados dos \textit{jobs} 
        e sinalizando que o experimento começou;
       \subitem A cada novo \textit{job} que deve começar (mudança de frequência da mesa) a aplicação web envia uma requisição
       de solicitação de alteração de frequência da mesa, seguindo o protocolo definido.
 \item O Servidor de Controle, então, envia uma mensagem ao sistema de controle (via rotinas de comunicação,
       que serão apresentadas na subseção \ref{software:rotinas}) informando o início do experimento e a frequência
       desejada (frequência do \textit{job} corrente) e inicia uma \textit{thread} para a coleta dos dados lidos dos sensores;
 \item Quando o último \textit{job} chegar ao fim, a aplicação web envia uma requisição ao Servidor de Controle informando
       que o ensaio acabou, o que significa que o sistema de controle não precisa mais coletar dados dos sensores;
 \item Quando o Servidor de Controle recebe essa requisição, este envia uma mensagem ao sistema de controle (via rotinas de comunicação)
       informando a parada da coleta dos dados dos sensores, seguindo o protocolo definido;
 \item A aplicação web informa ao usuário que está aguardando o término da coleta e processamento dos dados enquanto o Servidor de
       Controle coleta o restante dos dados dos sensores e a aplicação web processa os dados coletados;
 \item Quando todos os dados amostrados forem coletados, o Servidor de Controle repassa esses dados para a aplicação web e
       destes serão derivados os outros dados necessários para a aplicação, seguindo 
       o método apresentado na seção \ref{software:processamento}. O processamento dos dados aqui mencionado se refere aos cálculos da frequência,
       velocidade e amplitude a partir da aceleração fornecida pelos sensores;
 \item Com todos os dados calculados, a aplicação web monta os gráficos e apresenta os resultados do ensaio ao usuário.
\end{enumerate}


Nas próximas subseções são detalhados os dois subsistemas propostos como solução para o módulo de interface de processamento.


\subsection{Projeto e Construção}

  Nesta subseção é apresentado o projeto e construção dos subsistemas propostos na Subseção \ref{software:detalhamento_solucao},
  bem como as mudanças realizadas no que foi proposto.

\subsubsection{\textbf{Interface de comunicação com o usuário - Aplicação Web}} \label{software:app_web}
    
    Como  apresentado no \href{https://drive.google.com/file/d/0B5InkGKx6O-MR1B3eVYzZFpjQ3c/view?usp=sharing}{Relatório 1},
    a aplicação web servirá para o usuário controlar a bancada e visualizar os resultados dos ensaios, contando com os seguintes requisitos
    funcionais (conforme \textit{backlog} apresentado no \href{https://drive.google.com/file/d/0B5InkGKx6O-MR1B3eVYzZFpjQ3c/view?usp=sharing}{Relatório 1})
    para atender as necessidades do usuário (vide Apêndice \ref{documento_visao} para as necessidades do usuário):
    
    \begin{itemize}
      \item \textbf{Cadastro de usuários;}
	 \subitem Os usuários da bancada (alunos, professores e técnicos) devem ser cadastrados no sistema para utilizar a bancada.
      \item \textbf{Acesso ao sistema;}
	 \subitem Os usuários cadastrados devem ter acesso ao sistema para realizar ensaios.
      \item \textbf{Início do ensaio;}
	 \subitem Deve ser possível a criação de um ensaio específico pelo usuário, informando a frequência e tempo de ensaio desejados.
      \item \textbf{Visualização dos ensaios realizados;}
	 \subitem Os ensaios de um usuário devem ser mantidos no sistema, para que o usuário possa visualizar os resultados de um ensaio
		  já realizado quando quiser;
      \item \textbf{Visualização dos resultados do ensaio;}
	 \subitem Os resultados do ensaio realizado deve ser apresentado em formas de gráficos para facilitar a interpretação dos resultados.
    \end{itemize}
     
    Para o requisito "Início do ensaio" o usuário necessita poder criar um ensaio com diferentes frequências a serem testadas no mesmo ensaio.
    Para implementar este requisito foi realizada um abstração do ensaio como um conjunto de \textit{jobs}.
    Um \textit{job} é uma parte do ensaio com uma frequência e tempo fixo. Exemplificando, caso o usuário queira um ensaio que começa com uma
    frequência de 50Hz por 10s e depois mude para 80Hz e permaneça por mais 15s, basta criar um ensaio com dois \textit{jobs}, onde o primeiro
    \textit{job} teria frequência igual a 50Hz e tempo de 10s e o segundo teria frequência igual a 80Hz e tempo de 15s.
    
    Com o intuito de visualizar a solução Web foi desenhado um protótipo de alta fidelidade da aplicação Web. 
    As telas podem ser vistas no \href{https://drive.google.com/file/d/0B5InkGKx6O-MR1B3eVYzZFpjQ3c/view?usp=sharing}{Relatório 1}.
    
    Como definido no \href{https://drive.google.com/file/d/0B5InkGKx6O-MR1B3eVYzZFpjQ3c/view?usp=sharing}{Relatório 1},
    a aplicação web foi desenvolvida utilizando o \textit{framework} Django (em linguagem Python) de acordo com as justificativas
    apresentadas também no Relatório 1. 

\subsubsection{\textbf{Interface de comunicação com o sistema de controle - Servidor de Controle}}
   
   Esta subseção apresenta o projeto e construção do servidor RESTFul que será hospedado na \textit{Raspberry} para intermediar a comunicação
   com o sistema de controle. Este servidor é chamado aqui de Servidor de Controle.
  
\subsubsubsection*{\textbf{Mudanças - Aspectos gerais}}
    
Ao final do primeiro ponto de controle, a frente de interface/processamento, em suma, havia estabelecido os seguintes requisitos não-funcionais para o \textbf{software}:

\begin{itemize}
    \item Aplicação \textit{Web} para uso por parte do usuário, sendo esta desenvolvida com o \textit{framework} Django (que utiliza linguagem de programação \textit{Python}) juntamente com o SGBD (Sistema Gerenciador de Banco de Dados) \textit{MySQL}.
    \item Aplicação \textit{REST} para recebimento de comandos e fornecimento de dados para a Aplicação \textit{Web}, sendo esta desenvolvida com o Django \textit{REST} \textit{Framework} juntamente com a base de dados \textit{SQLite}.
    \item Rotinas de comunicação em linguagem C para escrita e leitura de dados na porta serial existente na \textit{Raspberry}.
\end{itemize}


Durante a execução do projeto, constatou-se que não haveria mais necessidade de uso da linguagem C, aspecto que será melhor detalhado na
próxima subseção, sobre as mudanças no projeto das rotinas de comunicações.
Outro aspecto alterado foi a base de dados utilizada pelo \textit{REST}, optando-se pelo uso do \textit{MySQL} ao invés do \textit{SQLite}.

O SGBD \textit{MySQL} foi também adotado para a aplicação \textit{REST} pelos seguintes motivos:

\begin{itemize}
  \item É mais performático que o \textit{SQLite}.
  \item Não há travamentos da base de dados devido às operações, o que possibilita a concorrência no uso do banco de dados.
  \item Favorece a escalabilidade da solução.
\end{itemize}

\subsubsubsection*{\textbf{Arquitetura das Rotinas de Comunicação e Parser}} \label{software:rotinas}

Para realização das rotinas de comunicação e parser, a equipe de software contou com um simulador. O simulador, construído pela frente de eletroeletrônica, 
consiste em um \textit{Arduino} que simula o comportamento da entrada e saída de dados que o sistema da \textit{Raspberry} terá de comunicar. Com isso 
então, foi possível realizar o desenvolvimento das rotinas de comunicação em \textit{Python}, bem como iniciar também o \textit{Parser}.

O \textit{Parser} foi construído utilizando a base de dados que o próprio \textit{Django} oferece, que no caso é o \textit{SQLITE3}, foi simulado uma 
entrada de uma \textit{String} seguindo o protocolo de comunicação e em seguida essa \textit{String} é tratada para que seja guardada no banco de dados 
da aplicação local. O primeiro passo consiste em conectar ao banco de dados, e em seguida realizar o tratamento dos dados em uma lista de forma a 
averiguar se os dados estão corretos de acordo com a entrada de elementos das tabelas no banco de dados. O próximo passo então insere os dados da lista 
no banco de dados, usando o método \textit{inserting acceleration} e assim por diante para os outros dados.

É importante ressaltar que o processamento está planejado para o ponto de controle 3, que utilizará essa base de dados local para realizar os cálculos 
e então inserir os dados de acordo com as tabelas de velocidade, amplitude e frequência, que serão utilizados pela aplicação web.

Com relação às rotinas de entrada e saída de dados, foi utilizado o \textit{Arduino} para fins de teste de comunicação. A frente de eletroeletrônica 
desenvolveu uma rotina que é executada no \textit{Arduino} que conseguia simular a entrada e saída de dados com o comportamento similar ao esperado. 
Então utilizado a biblioteca serial que é padrão em \textit{Python}, foi possível construir métodos abstraindo o controle da comunicação serial em baixo
nível para um alto nível, e conectá-los as rotinas que guardarão os dados, no caso o \textit{Parser}.

Para comunicação com a malha fechada de controle do sistema, foi desenvolvido o \textit{Pyserial}. Uma classe que abstrai os principais métodos da 
biblioteca padrão do \textit{Python} chamada serial. Com essa biblioteca foi possível inicializar os valores como \textit{BAUDRATE} (Taxa de transmissão),
\textit{Timeout} e a porta serial de entrada e saída dos dados. A classe também conta com um método simples que acumula os resultados de leitura da porta 
serial em uma grande lista que será utilizada pelo \textit{Parser} para o tratamento dos dados resultantes para que eles possam ser guardados no banco de 
dados. A classe também conta com um tratamento de exceções para evitar que as exceções subam em nível de usuário.

O \textit{RoutinesUtil}, é uma classe que faz parte da solução em \textit{Django} no pacote de classes e ele é responsável por definir o comportamento 
da rotina. Nesta classe são definidas as rotinas de leitura de dados resultantes, dos sensores ativos e da escrita de controle do sistema. Ela utiliza 
as classes \textit{Pyserial} e \textit{Parser} para abrir a porta serial, ler os dados, fechar a porta serial e gravar no banco de dados da aplicação do 
servidor, que mandará para aplicação Web.

A integração entre eletroeletrônica e software vem com as rotinas que irão se comunicar com a malha fechada de controle do sistema e com a 
\textit{Raspberry} da aplicação, responsáveis por armazenar os resultados e processá-las para a aplicação web. Na imagem abaixo você poderá verificar 
o diagrama de classes parcial das rotinas de comunicação em integração com o \textit{Django} da \textit{Raspberry} e seus respectivos métodos.

\begin{figure}[H]
\centering
\includegraphics[keepaspectratio=true,scale=0.65]{figuras/uml_routines_parser.png}
\label{fig:uml_routines_parser}
\caption{Diagrama de Classes Parcial das Rotinas de Comunicação e Parser, integrados ao \textit{Django} - Fonte: Autor}
\end{figure}


\subsubsubsection*{\textbf{Mudanças - Rotinas de Comunicação}}

A camada de software na aplicação Web foi desenvolvida em \textit{Python}, bem como a camada de servidor de comunicação com a aplicação na 
\textit{Raspberry}. Como foi relatado, os membros do grupo naquele momento haviam chegado ao consenso de utilizar a linguagem C para integrar 
com \textit{Python}, que seria utilizado nas rotinas de comunicação entre \textit{Raspberry} e os microcontroladores da aplicação de baixo nível.

Durante o desenvolvimento das rotinas de comunicação, percebeu-se que não havia necessidade de que as mesmas fossem desenvolvidas 
usando tal linguagem, considerando que a linguagem C é uma linguagem de baixo nível normalmente utilizada para ter um maior controle sobre o 
hardware, o que não é o caso do projeto. Levando em consideração este fato, foi levantada a hipótese de utilização do módulo de comunicação serial que o próprio 
\textit{Python} oferece, e para isso foi realizado uma lista de prós e contras acerca da linguagem no desenvolvimento das rotinas de comunicação.

Utilizando a linguagem C para o desenvolvimento das rotinas de comunicação possui as seguintes vantagens:

\begin{itemize}
    \item A linguagem C é uma linguagem de baixo nível, ou seja, oferece um maior controle do hardware, e fornece a possibilidade de manipulações com \textit{bytes} em uma maior flexibilidade do que com as demais linguagens, exceto a linguagem \textit{ASSEMBLY}.
    \item Com a linguagem C é possível desenvolver o sistema de forma a se proteger contra falhas críticas, impedindo de repassar erros de baixo nível para a camada de alto nível.
    \item A linguagem C realiza processamentos mais velozes em comparação com as outras linguagens como \textit{Python}, dentre outras, se tornando uma excelente ferramenta para manipulação de \textit{bytes}.
\end{itemize}

Desvantagens da utilização da linguagem C para o desenvolvimento das rotinas de comunicação:

\begin{itemize}
    \item Por ser uma linguagem de baixo nível com uma alta tipagem, o trabalho de desenvolver uma rotina de comunicação serial usando recursos da camada de hardware se tornam bastante complexos.
    \item Na arquitetura proposta no primeiro ponto de controle, as rotinas de escrita e leitura de dados funcionariam da seguinte maneira: a escrita iria ler um arquivo, em seguida iria apagá-lo e então, enviaria este dado para o sistema de controle; no sistema de leitura, seria aberta a porta seria para leitura dos \textit{bytes}, que seriam decodificados e transcritos num arquivo para serem lidos por uma rotina \textit{Python} para escrever no banco de dados. Ou seja, a integração possui uma alta complexidade.
    \item Apesar da linguagem C ser mais rápida que \textit{Python}, não existe necessidade das rotinas serem desenvolvidas em tal linguagem, pois a diferença de velocidade de escrita e leitura para comunicação serial é bem mínima levando em consideração ao que \textit{Python} também faz.
\end{itemize}

Vantagens em utilizar \textit{Python} para o desenvolvimento das rotinas de comunicação:

\begin{itemize}
    \item Python possui uma baixa tipagem.
    \item Possui uma biblioteca de comunicação serial que realiza a abstração do jeito que o desenvolvedor quiser (seja controlar em baixo nível, ou em alto nível com uma abstração do controle).
    \item Fácil integração com \textit{RESTful} do \textit{Django} que é realizado em \textit{Python}.
    \item Elimina necessidade de escrever e ler em arquivos, pois os métodos podem ser carregados pelo \textit{REST} diretamente, assim o \textit{REST} terá o controle total da comunicação.
    \item Não existe necessidade de manipular leitura e escrita de bytes, devida a abstração da biblioteca.
    \item \textit{Python} é uma linguagem melhor para manipulação de dados, e permitirá a escrita dos dados recebidos diretamente no banco de dados do servidor da \textit{Raspberry}, que a aplicação irá utilizar.
\end{itemize}

Desvantagens da linguagem \textit{Python} para o desenvolvimento das rotinas de comunicação:

\begin{itemize}
    \item \textit{Python} é uma linguagem interpretada e mais lenta em comparação à linguagem C.
    \item Não fornece um controle maior em comparação às linguagens de baixo nível.
\end{itemize}

Com a nova mudança então, a única linguagem de programação que está sendo utilizada neste projeto é a linguagem \textit{Python}, e em virtude 
desta mudança, a arquitetura mudou um pouco agora, sendo nesse novo estilo.

\begin{figure}[H]
\centering
\includegraphics[keepaspectratio=true,scale=0.7]{figuras/nova_arquitetura.png}
\label{fig:nova_arquitetura}
\caption{Nova arquitetura com a mudança da linguagem implementada - Fonte: Autor}
\end{figure}

\subsubsection*{\textbf{Protocolo de Comunicação}} \label{software:protocolo}

As rotinas de comunicação são as pontes pelas quais os componentes de software do servidor que estará rodando na \textit{Raspberry} irão se
comunicar com os componentes eletroeletrônicos do sistema de controle e, consequentemente, com o resto do sistema. 
Para que a comunicação seja estabelecida é necessário estabelecer um padrão de comunicação de acordo com ambas as frentes, de forma que a comunicação 
ocorra com sucesso e sem surpresas. Foi definido então um protocolo de comunicação que deve ser respeitado pela frente de software e
eletrônica durante a transmissão de dados via serial.

Com a mudança da arquitetura ficou mais simples a integração das rotinas de comunicação e de escrita (com o parser dos dados) com o
servidor, o que nos permitiu formalizar um protocolo bem simples para a comunicação entre o sistema de controle e o sistema da aplicação.

O protocolo de comunicação é dividido em duas partes: o protocolo de controle e o protocolo de resposta, que são apresentados abaixo.

\begin{itemize}
    \item \textbf{Protocolo de Controle}
    \begin{itemize}
        \item É definido por \textit{flags} que permitirão que o sistema realize ações pré-definidas.
    \end{itemize}
    \item \textbf{Protocolo de Resposta}
    \begin{itemize}
        \item É definida a \textit{STRING} de resposta que o sistema de controle enviará para a camada de alto nível, no caso a \textit{Raspberry}, via 
        comunicação serial.
    \end{itemize}
\end{itemize}

\begin{figure}[H]
\centering
\includegraphics[keepaspectratio=true,scale=0.7]{figuras/protocolo_controle.png}
\label{fig:protocolo_controle}
\caption{Protocolo de Controle com \textit{flags} já definidas - Fonte: Autor}
\end{figure}

\begin{figure}[H]
\centering
\includegraphics[keepaspectratio=true,scale=0.8]{figuras/protocolo_string.png}
\label{fig:protocolo_string}
\caption{Protocolo de Resposta e o formato da String que chegará ao sistema - Fonte: Autor}
\end{figure}

Com esse padrão definido, o usuário definirá um determinado número de \textit{jobs} (cada \textit{job} representa uma frequência com uma duração de tempo, 
e um conjunto de \textit{jobs} define um ensaio) que serão executados ao longo do ensaio. Ao executar o ensaio, os resultados dos sensores irão ser 
repassados na escala entre 0 a 1024 em AD (\textit{Analogic Digital}), sendo 0AD igual a 0G e 1024AD igual a 16G. Para determinar a aceleração a 
partir dessa escala basta utilizar uma regra de três simples. Por fim, esses dados passarão pelo processamento para serem persistidos no
banco de dados da \textit{Raspberry}.

Um dos objetivos ao obter a \textit{timestamp} a qual foi obtido o resultado pelo sensor é a necessidade de identificar e mapear os
\textit{jobs} de acordo com o ensaio, para ordenar e saber quais dados pertencem a quais \textit{jobs}.

A Figura \ref{fig:diagrama_sequencia_pc2} mostra um diagrama de sequência que apresenta a interação entre usuário, a aplicação web e o servidor da \textit{raspberry},
ilustrando como é tratado o \textit{input} e o \textit{output} do sistema como um todo entre os componentes de software e de eletrônica.

\begin{figure}[H]
\label{fig:diagrama_sequencia_pc2}
\centering
\includegraphics[keepaspectratio=true,scale=0.5]{figuras/diagrama_sequencia_pc2.png}
\caption{Diagrama de sequência - Fonte: Autor}
\end{figure}

\subsubsection*{\textbf{Processamento dos Dados}} \label{software:processamento}

Como mencionado, o processamento dos dados se refere ao cálculo da frequência, velocidade e amplitude a partir dos dados de aceleração
coletados dos sensores. Como os dados que se tem são coordenadas de aceleração por tempo, é preciso recorrer a integração numérica para 
cálcular esses dados, pois a velocidade (v(t)) é a integral da aceleração (a(t)) e o deslocamento (s(t)) é a integral da velocidade, como 
ilustram as fórmulas abaixo. A curva do deslocamento, neste caso, representa a amplitude (A) ao longo do tempo, pois o deslocamento calculado
da mesa, a partir dos dados dos sensores, representa o quanto a mesa subiu ou desceu (pois a mesa se move apenas no eixo Y).

$$ v(t) = \int_{}^{} a(t) dt$$
$$ A(t) = s(t) = \int_{}^{} v(t) dt$$

\subsubsubsection*{\textbf{Métodos numéricos analisados}}

A integral é uma das mais importantes ferramentas matemáticas e que aparece com frequência na solução de problemas e no cálculo de grandezas 
na engenharia e na ciência \cite{metodos_numericos}.

Na engenharia, há situações que envolvem dados experimentais ou de teste, nos quais uma grandeza física a ser determinada pode ser expressa 
como a integral de outras grandezas medidas. Assim, é válido ressaltar que o integrando pode ser uma função analítica ou um conjunto de pontos
discretos (dados tabulados).

Quando se tem um integrando expresso de forma que a integral pode ser facilmente calculada, pode-se obter analiticamente o valor da integral 
definida. A integração numérica faz-se necessária quando a integração analítica é difícil, ou mesmo impossível, e quando o integrando é fornecido
como um conjunto discreto de pontos \cite{metodos_numericos}.

Este é o caso da Bancada de Vibrações Mecânicas que está sendo construída neste projeto. Os dados que se tem são o retorno da leitura dos sensores 
e assim, o integrando é fornecido como um conjunto de pontos.

A análise numérica de uma integral caracteriza-se por estimar o número correspondente à integral de uma função entre os limites [a, b]. 
Caso seja utilizado apenas os pontos finais do intervalo, pode ser que o resultado fornecido não seja suficientemente preciso, especialmente se
o intervalo for significativamente grande ou se o integrando variar significativamente ao longo do intervalo. Dessa maneira, uma maior precisão 
pode ser obtida com o uso de um método composto, no qual o intervalo [a, b] é dividido em subintervalos menores. Assim, calcula-se a integral ao
longo de cada subintervalo e os resultados são somados para fornecer a integral completa.

É importante ressaltar que existem vários métodos disponíveis para o cálculo numérico de integrais. Esses métodos podem ser divididos em abertos 
e fechados. Os métodos de integração fechados consideram os pontos finais do intervalo e o integrando propriamente dito na fórmula que estima o
valor da integral. Já nos métodos de integração abertos, o intervalo de integração se estende além do limite especificado pelos pontos finais 
\cite{metodos_numericos}.

Devido à variedade de métodos, os principais foram analisados e verificou-se sua aplicação no âmbito do projeto. Os métodos considerados foram: 
Trapezoidal Composto, Métodos de Simpson e Quadratura de Gauss.

No {\textbf{Método Trapezoidal Composto}} a integral, ao longo do intervalo [a, b], pode ser avaliada de forma mais precisa com a subdivisão do 
intervalo, a avaliação da integral em cada um dos subintervalos e a soma dos resultados. A imagem a seguir ilustra a expansão genérica para a 
integração numérica por trapézios.

\begin{figure}[H]
\centering
\includegraphics[keepaspectratio=true,scale=0.52]	{figuras/metodo_trapezoidal.png}
\label{fig:metodo_trapezoidal}
\caption{Integração numérica por Trapézios Acumulados - Fonte: \citeonline{metodos_numericos}}
\end{figure}

É importante notar que a fórmula do método trapezoidal composto se aplica precisamente para os casos onde os subintervalos têm uma largura h
idêntica. Outro aspecto interessante no uso desse método é que a precisão da resposta pode melhorar a partir da utilização de mais subintervalos.

O método trapezoidal aproxima o integrando por uma linha reta. De fato, seria melhor a aproximação por meio da representação do integrando 
como uma função não-linear e de fácil integração.

Há um grupo de métodos dotados desta característica, denominados \textbf{Métodos de Simpson}, utilizando polinômios quadráticos (método de
Simpson 1/3) e polinômios cúbicos, no caso do método de Simpson 3/8.

Os métodos de Simpson possuem restrições mais severas para uso. No caso do método de Simpson 1/3 composto, os subintervalos devem ser 
igualmente espaçados e o número de subintervalos no intervalo [a, b] deve ser um número necessariamente par. Com relação ao método de Simpson 
3/8 composto, além do espaçamento igual na identificação dos subintervalos, tem-se que o número de subintervalos no intervalo [a, b] deve ser 
divisível por 3.

Por fim, na \textbf{Quadratura de Gauss}, a integral também é avaliada utilizando a soma ponderada dos valores da função em pontos distintos 
ao longo do intervalo [a, b]. Nessa abordagem, são utilizados os pontos de Gauss, que, por sua vez, não são igualmente espaçados e não incluem 
os pontos finais.

Para o projeto da Bancada de Vibrações, optou-se pelo uso do método dos trapézios acumulados (método trapezoidal composto). A frente de 
controle projetou uma leitura dos dados dos sensores que garantirá intervalos de tempo igualmente espaçados, já favorecendo a aplicação de 
tal método. Outro aspecto interessante é que não será necessário controlar se o número de subintervalos é par ou se é divisível por três. 
O método dos trapézios se aplica independentemente deste aspecto.

Outro aspecto que deve ser notado é que como a massa de dados coletada a partir da leitura das medições feitas pelos sensores é grande, a
precisão do método será ainda maior.

Para implementação do método dos trapézios, por meio de pesquisa, foi encontrada a Biblioteca \textit{SciPy}, disponível em \href{https://www.scipy.org/}{https://www.scipy.org/}. A mesma é implementada em linguagem \textit{Python} e adota a política de código aberto. A partir de uma rápida utilização da mesma, foi possível contemplar diversas ferramentas matemáticas amplamente utilizadas para resolução de problemas da ciência e engenharia.


\subsubsection{\textbf{Alternativas propostas e testes realizados}}


FALAR DOS TESTES DE CARGA REALIZADOS COM A SIMULAÇÃO


\section{Normas Técnicas e de Segurança}
Para garantir a integridade e segurança do projeto e dos futuros usuários do produto, deve-se atentar a determinadas normas técnicas, que garantem a funcionalidade do projeto. Essas normas ajudam a definir diversas características de um produto, visando uma padronização que ajuda na solução de futuros problemas, assegurando a qualidade, segurança e eficiência desejada.
	No Brasil, a Agência Brasileira de Normas Técnicas (ABNT) é o órgão responsável pela elaboração das Normas Técnicas Brasileiras (NBR), que servem como instrumento na avaliação de conformidades e certificação de diversas áreas e produtos, garantindo a devida credibilidade. Para o caso da Bancada de Vibração, alguns itens importantes podem ser destacados diante da NBR NM 213, que é uma norma específica voltada para Princípio de Projeto e Segurança de Máquinas e Equipamentos Elétricos.
    
\begin{table}[]
\centering
\caption{My caption}
\label{my-label}
\begin{tabular}{|c|c|c|}
\hline
\textbf{ITEM AVALIADO}                                                                           & \textbf{MEDIDA NORMATIVA}                                                                                                                                                                      & \textbf{O QUE SERÁ FEITO NA BANCADA?}                                                                                                                                                                                                                                                                                     \\ \hline
\begin{tabular}[c]{@{}c@{}}Arestas vivas, ângulos vivos, peças salientes,\\ 			etc.\end{tabular} & Verificar a uniformidade e as quinas da bancada.                                                                                                                                               & \begin{tabular}[c]{@{}c@{}}Será utilizado um material uniforme como tampo da\\ 			bancada e serão colocados protetores de quina.\end{tabular}                                                                                                                                                                             \\ \hline
Formas e posição de componentes mecânicos.                                                       & \begin{tabular}[c]{@{}c@{}}Previnir, durante a operação da bancada, o\\ 			acesso do usuário aos componentes de risco.\end{tabular}                                                            & \begin{tabular}[c]{@{}c@{}}Durante o projeto da estrutura, os motores e a\\ 			polia serão dispostos à uma distância segura da posição de\\ 			operação.\end{tabular}                                                                                                                                                     \\ \hline
Ergonomia                                                                                        & \begin{tabular}[c]{@{}c@{}}Reduzir as tensões nervosas e o esforço físico\\ 			do operador\end{tabular}                                                                                        & \begin{tabular}[c]{@{}c@{}}Garantir uma altura da bancada que não\\ 			sobrecarregue o usuário, fisicamente, na operação da bancada.\\ 			Partindo de uma altura média esperada pelo usuário.\end{tabular}                                                                                                                \\ \hline
Sistema de Comando                                                                               & \begin{tabular}[c]{@{}c@{}}Garantir a fácil\\ 			intervenção do operador em casos de:\\ 			- Partida inesperada\\ 			- Velocidade não\\ 			controlada\\ 			- Pesos insustentáveis\end{tabular} & \begin{tabular}[c]{@{}c@{}}O sistema da bancada terá “faixas” de\\ 			frequência e velocidade acessíveis ao usuário, não permitindo\\ 			que tais valores sejam extrapolados. Uma célula de carga será\\ 			acoplada para informar se o objeto testado está acima do peso\\ 			suportado pela estrutura.\end{tabular}     \\ \hline
Vigilância Automática                                                                            & Desencadear uma ação de segurança                                                                                                                                                              & \begin{tabular}[c]{@{}c@{}}Será implantado uma parada automática e/ou\\ 			alarmes em comportamentos inesperados\end{tabular}                                                                                                                                                                                             \\ \hline
Perigo Elétrico                                                                                  & \begin{tabular}[c]{@{}c@{}}Proteções contra:\\ 			- choques elétricos\\ 			- curtos circuitos\\ 			- sobrecargas\end{tabular}                                                                  & \begin{tabular}[c]{@{}c@{}}Haverá um aviso na bancada sobre a fonte de\\ 			alimentação que deve ser usada e um relé de proteção para\\ 			preservar a integridade dos motores e dos circuitos. Ao longo da\\ 			bancada, nos pontos que forem determinados críticos, será\\ 			utilizado material isolante.\end{tabular} \\ \hline
Estrutura                                                                                        & Robustez                                                                                                                                                                                       & \begin{tabular}[c]{@{}c@{}}A estrutura será projetada e testada para que a\\ 			bancada não fique instável e nem balance, fixando os motores,\\ 			polias e pés\end{tabular}                                                                                                                                              \\ \hline
Exigências Gerais, avisos, inscrições.                                                           & \begin{tabular}[c]{@{}c@{}}Deve existir clareza nas informações de\\ 			operação.\end{tabular}                                                                                                 & \begin{tabular}[c]{@{}c@{}}Haverá um/uns adesivo/adesivos na bancada\\ 			informando sobre procedimentos de operação, instruções\\ 			técnicas e prevenções de riscos. Tais avisos serão localizados\\ 			em partes estratégicas da bancada\end{tabular}                                                                  \\ \hline
Documento de acompanhamento                                                                      & Manual de instrução                                                                                                                                                                            & \begin{tabular}[c]{@{}c@{}}Será elaborado um manual abordando os itens de\\ 			segurança, utilização, condição de armazenamento, indicações\\ 			para a movimentação, dimensões, exigências de espaço físico,\\ 			regulagens e ajustes, manutenções, situações de emergência,\\ 			etc.\end{tabular}                     \\ \hline
Medidas adicionais                                                                               & Medidas previstas para situações de emergência                                                                                                                                                 & \begin{tabular}[c]{@{}c@{}}Será implementado um dispositivo de parada de\\ 			emergência,\end{tabular}                                                                                                                                                                                                                    \\ \hline
Manutenção                                                                                       & \begin{tabular}[c]{@{}c@{}}Considerar fatores que facilitem a\\ 			manutenibilidade do sistema.\end{tabular}                                                                                   & \begin{tabular}[c]{@{}c@{}}A estrutura será projetada atentando ao fácil\\ 			acesso de partes internas.\end{tabular}                                                                                                                                                                                                     \\ \hline
\end{tabular}
\end{table}